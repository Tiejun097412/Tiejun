% !Mode:: "TeX:UTF-8"
\documentclass[12pt,a4paper]{article}
\input{en_preamble.tex}
\input{xecjk_preamble.tex}

\title{梯度下降法}
\author{作者:周铁军}
\date{\chntoday}
\begin{document}
\maketitle
\newpage
\subsection{描述}
梯度下降法基于以下的观察:如果实值函数$F(x)$在点a处可微且有定义,那么函数$F(x)$
在a点沿着梯度相反的方向$-\nabla F(a)$下降最快。

因而,如果
\begin{equation}
	b=a-\gamma\nabla F(a)
\end{equation}
对于$\gamma>0$为一个够小数值时成立,那么$F(a)\ge F(b)$。

考虑到这一点,我们可以从函数F的局部极小值的初始估计$x_0$出发,并考虑如下序列
$x_0,x_1,x_2,...$使得
\begin{equation}
	x_{n+1}=x_n-\gamma_n \nabla F(x_n),n \ge 0。
\end{equation}
因此可得到
\begin{equation}
	F(x_0) \ge F(x_1) \ge F(x_2) \ge ...,
\end{equation}
如果顺利的话序列$(x_n)$收敛到期望的极值。注意每次迭代步长$\gamma$
可以改变。

\begin{figure}[H]
\centering
\includegraphics[scale=0.5]{./figures/Figure_2.png}
\end{figure}
上图示例了这一过程,这里假设F定义在平面上,并且函数图像是一个碗形。蓝色的曲线是等高线(水平集),即函数 F为常数的集合构成的曲线。红色的箭头指向该点梯度的反方向。(一点处的梯度方向与通过该点的等高线垂直)。沿着梯度下降方向,将最终到达碗底,即函数 F值最小的点。

\subsection{例子}
梯度下降法处理一些复杂的非线性函数会出现问题,例如Rosenbrock函数
\begin{equation}
	f(x,y)=(1-x)^2+100(y-x^2)^2
\end{equation}
其最小值在(x,y)=(1,1)处,数值为f(x,y)=0.但是此函数具有狭窄弯的山谷,最小值 (x,y)=(1,1)就在这些山谷之中,并且谷底很平。优化过程是之字形的向极小值点靠近,速度非常缓慢。

\begin{figure}[H]
\centering
\includegraphics[scale=0.4]{./figures/Figure_3.png}
\end{figure}
下面这个例子也鲜明的示例了"之字"的上升(非下降),这个例子用梯度上升(非梯度下降)法求$ F(x,y)=\sin \left({\frac  {1}{2}}x^{2}-{\frac  {1}{4}}y^{2}+3\right)\cos(2x+1-e^{y})$的极大值(非极小值,实际是局部极大值)。


\begin{figure}[ht]
\begin{minipage}[t]{0.4\linewidth}%并排放两张图片,每张占行的0.4,下同 
\centering     %插入的图片居中表示
\includegraphics[width=1.2\textwidth]{./figures/Figure_4.png}
\caption{this is a figure4.}%图片的名称
\label{fig:liuchengtu1}%标签,用作
\end{minipage} 
\hfill
\begin{minipage}[t]{0.4\linewidth}
\centering
\includegraphics[width=1.2\textwidth]{./figures/Figure_5.png}
\caption{this is a figure5.}%图片的名称
\label{fig:liuchengtu2}
\end{minipage}
\end{figure}























\cite{tam19912d}
\bibliography{ref}
\end{document}
