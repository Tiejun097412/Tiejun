% !Mode:: "TeX:UTF-8"
\documentclass[12pt,a4paper]{article}
\input{en_preamble.tex}
\input{xecjk_preamble.tex}

\title{谱延迟校正}
\author{作者:周铁军}
\date{\chntoday}
\begin{document}
\maketitle
\newpage

\begin{equation}
	q(s,r)=q(0,r)+\int_{0}^{s}[\nabla^2 q(\tau,r)-\omega(r)q(\tau,r)]d\tau
\end{equation}

构造如下迭代格式:
\begin{equation}
	q^{[1]}(s,r)=q(0,r)+\int_{0}^{s}[\nabla^2 q^{[0]}q(\tau,r)-w(r)q^{[0]}(\tau,r)]d\tau
\end{equation}

对s所在区间$[0,f]$作$n$次剖分,节点记为$s_i,i=0,1,...,n$

易知,$q(s,r)=[q(s_0,r),q(s_1,r),...,q(s_n,r)]'$

取变量$x \in {s_0,s_1,...,s_n}$

我们可以得到$n+1$个式子
\begin{equation}
	q^{[1]}(x,r)=q(0,r)+\int_{0}^{x}[\nabla^2 q^{[0]}q(\tau,r)-w(r)q^{[0]}(\tau,r)]d\tau
\end{equation}

通过$\tau=\frac{x}{2}+\frac{x}{2} t$,$\tau \in (0,x)$映射到$t \in [-1,1]$

于是得到
\begin{equation}
	q^{[1]}(x,r)=q(0,r)+\int_{-1}^{1}[\nabla^2 q^{[0]}q(\frac{x}{2}+\frac{x}{2} t,r)-w(r)q^{[0]}(\frac{x}{2}+\frac{x}{2} t,r)]\frac{x}{2}dt
\end{equation}

现在$t\in[-1,1]$,$s_j=-cos(\frac{j\pi}{n})$是$[-1,1]$上的节点。

%%带边框 
 记:$$f(t,r)=[\nabla^2 q^{[0]}q(\frac{x}{2}+\frac{x}{2} t,r)-w(r)q^{[0]}(\frac{x}{2}+\frac{x}{2} t,r)]\frac{x}{2}$$
 现在要求$f(t,r)$的插值多项式,我们对r所在区间$[0,10]$作剖分,得$r_0,r_1,...,r_n$
 这样我们可得到一组关于$t$的一元函数$f(t,r_i),i=0,1,...,n$

作$f(t,r_i)$的插值逼近
\begin{equation}
	p_{n}(t)=\frac{a_0}{2}+\sum_{k=1}^{n-1}a_k T_k(t)+\frac{a_n}{2}T_n(t)
\end{equation}

用$t=-cos\theta,\theta \in [0,\pi]$带入得到
\begin{equation}
	p_n(-cos\theta)=\frac{a_0}{2}+\sum_{k=1}^{n-1}a_k \cos k\theta+\frac{a_n}{2}\cos n\theta
\end{equation}
该式是对
\begin{equation}
	F(\theta,r_i)=[f(-cos\theta,r_i)=\nabla^2 q^{[0]}(\frac{x}{2}+\frac{x}{2} \cos \theta,r)-w(r)q^{[0]}(\frac{x}{2}+\frac{x}{2}\cos \theta ,r)]\frac{x}{2}
\end{equation}
的插值逼近

















\cite{tam19912d}
\bibliography{ref}
\end{document}
