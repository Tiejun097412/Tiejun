% !Mode:: "TeX:UTF-8"
\documentclass[12pt,a4paper]{article}
\input{en_preamble.tex}
\input{xecjk_preamble.tex}

\title{五点差分求解possion方程}
\author{作者:周铁军}
\date{\chntoday}
\begin{document}
\maketitle
\newpage
\section{五点差分}
 题目:用差分法求解边值问题
 
 \begin{equation}
  \begin{array}{l}{-\Delta u = cos{3x}sin{\pi y} \qquad (x,y) \in G=(0,\pi)\times(0,1),}\\
  {u(x,0) = u(x,1)=0,\quad 0 \leq x \leq 1, }\\
  {u_x(0,y) = u_x(\pi,y)=0,\quad 0 \leq y \leq 1, }\end{array}
 \end{equation}
其中精确解为:$u = (9+\pi^2)^{-1}cos{3x}sin{\pi y}$

(1) 依次取 $N = 4,8,16,32$,取 $6$ 位小数计算,以步长$h_1 = \frac{\pi}{N}$和$\frac{1}{N}$作矩形剖分,就$(x_i,y_j)=(\frac{i\pi}{4},\frac{j}{4}),i,j=1,2,3$处列出差分解与精确解.

(2) 计算出差分法的误差阶.

\subsection{求解}
首先,对进行网格剖分,其中$h_1$和$h_2$分别为区间$[0,\pi]$和$[0,1]$上的剖分步长。

其次,利用五点差分离散二阶导数$\frac{\partial^2 u}{\partial x^2}$和$\frac{\partial^2 u}{\partial y^2}$,得到:
\begin{equation}
\begin{array}{l}{\frac{\partial^2 u}{\partial x^2} = \frac{u_{i+1,j}+u_{i-1,j}-2u_{i,j}}{h_1^2} + O(h_1^2)}\\
{\frac{\partial^2 u}{\partial y^2} = \frac{u_{i,j+1}+u_{i,j-1}-2u_{i,j}}{h_2^2}+ O(h_2^2)}\end{array}
\end{equation}
其中$i,j=0,1,...,n$,于是方程可表示为:
$$-\frac{u_{i+1,j}+u_{i-1,j}-2u_{i,j}}{h_1^2}-\frac{u_{i,j+1}+u_{i,j-1}-2u_{i,j}}{h_2^2} = cos{3x_i}sin{\pi y_j},\quad i,j=0,1,...,n$$

化简得:
$$(\frac{2}{h_1^2}+\frac{2}{h_2^2})u_{i,j}-\frac{1}{h_1^2}(u_{i+1,j}+u_{i-1,j})-\frac{1}{h_2^2}(u_{i,j+1}+u_{i,j-1}) = cos{3x_i}sin{\pi y_j},\quad i,j=0,1,...,n$$


考虑边界条件,有:
\begin{equation}
\begin{array}{l}{u(x,y(0)) = u(x,y(n))=0}\\
{u(x(0),y) = u(x(1),y),u(x(n-1),y) = u(x(n),y)}\end{array}
\end{equation}
\newpage 
则只需求解红色区域内的点(见图1):
\begin{figure}[ht]
	\centering
	\includegraphics[scale=0.4]{./figures/zuobiao.png}
	\caption{网格剖分}
	\label{fig:label}	
\end{figure}



化为矩阵形式:
$$AU = F$$

其中$U = (u_{11},u_{12},...,u_{1,n-1},u_{21},u_{22},...,u_{2,n-1},...)^{T} \quad (U\text{为}(n-1)^2 \times 1 \text{向量})$

$F = (f_{11},f_{12},...,f_{1,n-1},f_{21},f_{22},...,f_{2,n-1},...)^{T} \quad (f(x,y)=cos{3x}sin{\pi y})$

\begin{equation}
A = \left[                 
\begin{array}{cccc}
B_1& C &\cdots  & 0\\
C& B_2 &\cdots& 0\\
\vdots  & \vdots & \ddots & \vdots \\
0& 0 &\cdots  &B_1
\end{array}
\right ]_{(n-1)^2 \times (n-1)^2},\quad B_1 = \left[                 
\begin{array}{cccc}
\frac{1}{h_1^2}+\frac{2}{h_2^2}& -\frac{1}{h_2^2} &\cdots  & 0\\
-\frac{1}{h_2^2}& \frac{1}{h_1^2}+\frac{2}{h_2^2} &\cdots& 0\\
\vdots  & \vdots & \ddots & \vdots \\
0& 0 &\cdots  &\frac{1}{h_1^2}+\frac{2}{h_2^2}
\end{array}
\right ]_{(n-1) \times (n-1)} ,
\end{equation}

\begin{equation}
 B_2 = \left[                 
\begin{array}{cccc}
\frac{2}{h_1^2}+\frac{2}{h_2^2}& -\frac{1}{h_2^2} &\cdots  & 0\\
-\frac{1}{h_2^2}& \frac{2}{h_1^2}+\frac{2}{h_2^2} &\cdots& 0\\
\vdots  & \vdots & \ddots & \vdots \\
0& 0 &\cdots  &\frac{2}{h_1^2}+\frac{2}{h_2^2}
\end{array}
\right ]_{(n-1) \times (n-1)} ,\quad C = -\frac{1}{h_1^2} I_{n-1}
\end{equation}

\subsection{结果}
下面展示$N = 4$时,$(x_i,y_j)=(\frac{i\pi}{4},\frac{j}{4}),i,j=1,2,3$处的差分解与精确解.
\begin{equation}
\text{差分解:}u = \left[                 
\begin{array}{ccc}
-0.042357& -0.059903 & -0.042357\\
-7.598053\times 10^{-18}& -1.123532\times 10^{-17} & -8.021530 \times 10^{-18}\\
0.042357& 0.059903 & 0.042357
\end{array}
\right ]
\end{equation}

\begin{equation}
\text{精确解:}ue = \left[                 
\begin{array}{ccc}
-0.026498& -0.037473 & -0.026498\\
-6.883738\times 10^{-18}& -9.735075\times 10^{-18} & -6.883738\times 10^{-18}\\
0.026498& 0.037473 & 0.026498
\end{array}
\right ]
\end{equation}
\newpage
其他剖分的解暂不列出,下面列出不同剖分下求得的误差范数。

\begin{table*}
	\centering  
	\caption{误差分析}  
	\begin{tabular*}{6cm}{lll}  
		\hline  
		N & max范数  & $L_2$范数 \\  
		\hline  
		4  & 0.0224 & 0.0449  \\  
		8  & 0.0156 & 0.0509  \\
		16  & 0.0229 & 0.1105 \\
		32  & 0.0321 & 0.2681 \\  
		\hline  
	\end{tabular*}  
\end{table*}

\subsection{我的问题}
通过数值求解,我得到的误差阶是一阶的,但是该问题本应该是二阶,

还请老师,师姐指教,matlab代码见附件。

\end{document}
