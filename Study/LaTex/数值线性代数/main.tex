% !Mode:: "TeX:UTF-8"
\documentclass[12pt,a4paper]{article}

%%%%%%%%------------------------------------------------------------------------
%%%% 日常所用宏包

%% 控制页边距
% 如果是beamer文档类, 则不用geometry
\makeatletter
\@ifclassloaded{beamer}{}{\usepackage[top=2.5cm, bottom=2.5cm, left=2.5cm, right=2.5cm]{geometry}}
\makeatother

%% 控制项目列表
\usepackage{enumerate}

%% 多栏显示
\usepackage{multicol}

%% 算法环境
\usepackage{algorithm}  
\usepackage{algorithmic} 
\usepackage{float} 

%% 网址引用
\usepackage{url}

%% 控制矩阵行距
\renewcommand\arraystretch{1.4}

%% hyperref宏包,生成可定位点击的超链接,并且会生成pdf书签
\makeatletter
\@ifclassloaded{beamer}{
\usepackage{hyperref}
\usepackage{ragged2e} % 对齐
}{
\usepackage[%
    pdfstartview=FitH,%
    CJKbookmarks=true,%
    bookmarks=true,%
    bookmarksnumbered=true,%
    bookmarksopen=true,%
    colorlinks=true,%
    citecolor=blue,%
    linkcolor=blue,%
    anchorcolor=green,%
    urlcolor=blue%
]{hyperref}
}
\makeatother



\makeatletter % 如果是 beamer 不需要下面两个包
\@ifclassloaded{beamer}{
\mode<presentation>
{
} 
}{
%% 控制标题
\usepackage{titlesec}
%% 控制目录
\usepackage{titletoc}
}
\makeatother

%% 控制表格样式
\usepackage{booktabs}

%% 控制字体大小
\usepackage{type1cm}

%% 首行缩进,用\noindent取消某段缩进
\usepackage{indentfirst}

%%边框
\usepackage{listings}

%% 支持彩色文本、底色、文本框等
\usepackage{color,xcolor}

%% AMS LaTeX宏包: http://zzg34b.w3.c361.com/package/maths.htm#amssymb
\usepackage{amsmath,amssymb}
%% 多个图形并排
\usepackage{subfig}
%%%% 基本插图方法
%% 图形宏包
\usepackage{graphicx}
\newcommand{\red}[1]{\textcolor{red}{#1}}
\newcommand{\blue}[1]{\structure{#1}}
\newcommand{\brown}[1]{\textcolor{brown}{#1}}
\newcommand{\green}[1]{\textcolor{green}{#1}}


%%%% 基本插图方法结束

%%%% pgf/tikz绘图宏包设置
\usepackage{pgf,tikz}
\usetikzlibrary{shapes,automata,snakes,backgrounds,arrows}
\usetikzlibrary{mindmap}
%% 可以直接在latex文档中使用graphviz/dot语言,
%% 也可以用dot2tex工具将dot文件转换成tex文件再include进来
%% \usepackage[shell,pgf,outputdir={docgraphs/}]{dot2texi}
%%%% pgf/tikz设置结束


\makeatletter % 如果是 beamer 不需要下面两个包
\@ifclassloaded{beamer}{

}{
%%%% fancyhdr设置页眉页脚
%% 页眉页脚宏包
\usepackage{fancyhdr}
%% 页眉页脚风格
\pagestyle{plain}
}

%% 有时会出现\headheight too small的warning
\setlength{\headheight}{15pt}

%% 清空当前页眉页脚的默认设置
%\fancyhf{}
%%%% fancyhdr设置结束


\makeatletter % 对 beamer 要重新设置
\@ifclassloaded{beamer}{

}{
%%%% 设置listings宏包用来粘贴源代码
%% 方便粘贴源代码,部分代码高亮功能
\usepackage{listings}

%% 设置listings宏包的一些全局样式
%% 参考http://hi.baidu.com/shawpinlee/blog/item/9ec431cbae28e41cbe09e6e4.html
\lstset{
showstringspaces=false,              %% 设定是否显示代码之间的空格符号
numbers=left,                        %% 在左边显示行号
numberstyle=\tiny,                   %% 设定行号字体的大小
basicstyle=\footnotesize,                    %% 设定字体大小\tiny, \small, \Large等等
keywordstyle=\color{blue!70}, commentstyle=\color{red!50!green!50!blue!50},
                                     %% 关键字高亮
frame=shadowbox,                     %% 给代码加框
rulesepcolor=\color{red!20!green!20!blue!20},
escapechar=`,                        %% 中文逃逸字符,用于中英混排
xleftmargin=2em,xrightmargin=2em, aboveskip=1em,
breaklines,                          %% 这条命令可以让LaTeX自动将长的代码行换行排版
extendedchars=false                  %% 这一条命令可以解决代码跨页时,章节标题,页眉等汉字不显示的问题
}}
\makeatother
%%%% listings宏包设置结束


%%%% 附录设置
\makeatletter % 对 beamer 要重新设置
\@ifclassloaded{beamer}{

}{
\usepackage[title,titletoc,header]{appendix}
}
\makeatother
%%%% 附录设置结束


%%%% 日常宏包设置结束
%%%%%%%%------------------------------------------------------------------------


%%%%%%%%------------------------------------------------------------------------
%%%% 英文字体设置结束
%% 这里可以加入自己的英文字体设置
%%%%%%%%------------------------------------------------------------------------

%%%%%%%%------------------------------------------------------------------------
%%%% 设置常用字体字号,与MS Word相对应

%% 一号, 1.4倍行距
\newcommand{\yihao}{\fontsize{26pt}{36pt}\selectfont}
%% 二号, 1.25倍行距
\newcommand{\erhao}{\fontsize{22pt}{28pt}\selectfont}
%% 小二, 单倍行距
\newcommand{\xiaoer}{\fontsize{18pt}{18pt}\selectfont}
%% 三号, 1.5倍行距
\newcommand{\sanhao}{\fontsize{16pt}{24pt}\selectfont}
%% 小三, 1.5倍行距
\newcommand{\xiaosan}{\fontsize{15pt}{22pt}\selectfont}
%% 四号, 1.5倍行距
\newcommand{\sihao}{\fontsize{14pt}{21pt}\selectfont}
%% 半四, 1.5倍行距
\newcommand{\bansi}{\fontsize{13pt}{19.5pt}\selectfont}
%% 小四, 1.5倍行距
\newcommand{\xiaosi}{\fontsize{12pt}{18pt}\selectfont}
%% 大五, 单倍行距
\newcommand{\dawu}{\fontsize{11pt}{11pt}\selectfont}
%% 五号, 单倍行距
\newcommand{\wuhao}{\fontsize{10.5pt}{10.5pt}\selectfont}
%%%%%%%%------------------------------------------------------------------------


%% 设定段间距
\setlength{\parskip}{0.5\baselineskip}

%% 设定行距
\linespread{1}


%% 设定正文字体大小
% \renewcommand{\normalsize}{\sihao}

%制作水印
\RequirePackage{draftcopy}
\draftcopyName{XTUMESH}{100}
\draftcopySetGrey{0.90}
\draftcopyPageTransform{40 rotate}
\draftcopyPageX{350}
\draftcopyPageY{80}

%%%% 个性设置结束
%%%%%%%%------------------------------------------------------------------------


%%%%%%%%------------------------------------------------------------------------
%%%% bibtex设置

%% 设定参考文献显示风格
% 下面是几种常见的样式
% * plain: 按字母的顺序排列,比较次序为作者、年度和标题
% * unsrt: 样式同plain,只是按照引用的先后排序
% * alpha: 用作者名首字母+年份后两位作标号,以字母顺序排序
% * abbrv: 类似plain,将月份全拼改为缩写,更显紧凑
% * apalike: 美国心理学学会期刊样式, 引用样式 [Tailper and Zang, 2006]

\makeatletter
\@ifclassloaded{beamer}{
\bibliographystyle{apalike}
}{
\bibliographystyle{unsrt}
}
\makeatother


%%%% bibtex设置结束
%%%%%%%%------------------------------------------------------------------------

%%%%%%%%------------------------------------------------------------------------
%%%% xeCJK相关宏包

\usepackage{xltxtra,fontspec,xunicode}
\usepackage[slantfont, boldfont]{xeCJK} 

\setlength{\parindent}{2em}%中文缩进两个汉字位

%% 针对中文进行断行
\XeTeXlinebreaklocale "zh"             

%% 给予TeX断行一定自由度
\XeTeXlinebreakskip = 0pt plus 1pt minus 0.1pt

%%%% xeCJK设置结束                                       
%%%%%%%%------------------------------------------------------------------------

%%%%%%%%------------------------------------------------------------------------
%%%% xeCJK字体设置

%% 设置中文标点样式,支持quanjiao、banjiao、kaiming等多种方式
\punctstyle{kaiming}                                        
                                                     
%% 设置缺省中文字体
%\setCJKmainfont[BoldFont={Adobe Heiti Std}, ItalicFont={Adobe Kaiti Std}]{Adobe Song Std}   
\setCJKmainfont{SimSun}
%% 设置中文无衬线字体
%\setCJKsansfont[BoldFont={Adobe Heiti Std}]{Adobe Kaiti Std}  
%% 设置等宽字体
%\setCJKmonofont{Adobe Heiti Std}                            

%% 英文衬线字体
\setmainfont{DejaVu Serif}                                  
%% 英文等宽字体
\setmonofont{DejaVu Sans Mono}                              
%% 英文无衬线字体
\setsansfont{DejaVu Sans}                                   

%% 定义新字体
\setCJKfamilyfont{song}{Adobe Song Std}                     
\setCJKfamilyfont{kai}{Adobe Kaiti Std}
\setCJKfamilyfont{hei}{Adobe Heiti Std}
\setCJKfamilyfont{fangsong}{Adobe Fangsong Std}
\setCJKfamilyfont{lisu}{LiSu}
\setCJKfamilyfont{youyuan}{YouYuan}

%% 自定义宋体
\newcommand{\song}{\CJKfamily{song}}                       
%% 自定义楷体
\newcommand{\kai}{\CJKfamily{kai}}                         
%% 自定义黑体
\newcommand{\hei}{\CJKfamily{hei}}                         
%% 自定义仿宋体
\newcommand{\fangsong}{\CJKfamily{fangsong}}               
%% 自定义隶书
\newcommand{\lisu}{\CJKfamily{lisu}}                       
%% 自定义幼圆
\newcommand{\youyuan}{\CJKfamily{youyuan}}                 

%%%% xeCJK字体设置结束
%%%%%%%%------------------------------------------------------------------------

%%%%%%%%------------------------------------------------------------------------
%%%% 一些关于中文文档的重定义
\newcommand{\chntoday}{\number\year\,年\,\number\month\,月\,\number\day\,日}
%% 数学公式定理的重定义

%% 中文破折号,据说来自清华模板
\newcommand{\pozhehao}{\kern0.3ex\rule[0.8ex]{2em}{0.1ex}\kern0.3ex}

\newtheorem{example}{例}                                   
\newtheorem{theorem}{定理}[section]                         
\newtheorem{definition}{定义}
\newtheorem{axiom}{公理}
\newtheorem{property}{性质}
\newtheorem{proposition}{命题}
\newtheorem{lemma}{引理}
\newtheorem{corollary}{推论}
\newtheorem{remark}{注解}
\newtheorem{condition}{条件}
\newtheorem{conclusion}{结论}
\newtheorem{assumption}{假设}

\makeatletter %
\@ifclassloaded{beamer}{

}{
%% 章节等名称重定义
\renewcommand{\contentsname}{目录}     
\renewcommand{\indexname}{索引}
\renewcommand{\listfigurename}{插图目录}
\renewcommand{\listtablename}{表格目录}
\renewcommand{\appendixname}{附录}
\renewcommand{\appendixpagename}{附录}
\renewcommand{\appendixtocname}{附录}
%% 设置chapter、section与subsection的格式
\titleformat{\chapter}{\centering\huge}{第\thechapter{}章}{1em}{\textbf}
\titleformat{\section}{\centering\sihao}{\thesection}{1em}{\textbf}
\titleformat{\subsection}{\xiaosi}{\thesubsection}{1em}{\textbf}
\titleformat{\subsubsection}{\xiaosi}{\thesubsubsection}{1em}{\textbf}

\@ifclassloaded{book}{

}{
\renewcommand{\abstractname}{摘要}
}
}
\makeatother

\renewcommand{\figurename}{图}
\renewcommand{\tablename}{表}

\makeatletter
\@ifclassloaded{book}{
\renewcommand{\bibname}{参考文献}
}{
\renewcommand{\refname}{参考文献} 
}
\makeatother

\floatname{algorithm}{算法}
\renewcommand{\algorithmicrequire}{\textbf{输入:}}
\renewcommand{\algorithmicensure}{\textbf{输出:}}

%%%% 中文重定义结束
%%%%%%%%------------------------------------------------------------------------


\title{矩阵计算}
\author{潘建瑜}

\begin{document}
\maketitle
\newpage
\section{课程主要内容}

• 线性方程组的直接解法

• 线性最小二乘问题的数值算法

• 非对称矩阵的特征值计算

• 对称矩阵特征值计算与奇异值分解

• 线性方程组迭代算法

• 特征值问题的迭代算法 (部分特征值和特征向量)

• 稀疏矩阵计算

\subsection*{主要参考资料}

• G. H. Golub and C. F. van Loan, “Matrix Computations (4th),” 2013.对应中文版为:《矩阵计算》(第三版), 袁亚湘等译, 2001.

• J. W. Demmel, “Applied Numerical Linear Algebra,” 1997.对应中文版为:《应用数值线性代数》, 王国荣译, 2007.

• L. N. Trefethen and D. Bau, III, “Numerical Linear Algebra,” 1997.对应中文版为:《数值线性代数》, 陆金甫等译, 2006.

• 徐树方, “矩阵计算的理论与方法,” 北京大学出版社, 1995.

• 曹志浩, “数值线性代数,” 复旦大学出版社, 1996.

\subsection*{课程主页}

http://math.ecnu.edu.cn/~jypan/Teaching/MatrixComp/

\newpage
\section{引言}
计算数学, 也称 数值分析 或 计算方法.

1947 年 Von Neumann 和 Goldstine 在《美国数学会通报》发表了题为“高阶矩阵的数值求逆”的著名论文, 开启了现代计算数学的研究

一般来说, 计算数学主要研究如何求出数学问题的近似解 (数值解), 包括算法的设计与分析 (收敛性, 稳定性, 复杂性等)

\begin{framed}
	
	计算数学主要研究内容:
	
	数值逼近, 数值微积分, 数值代数, 微分方程数值解, 最优化等

\end{framed}

\subsection{为什么学计算数学}
科学计算是 20 世纪重要科学技术进步之一, 已与理论研究和实验研究相并列成为科学研究的第三种方法. 现今科学计算已是体现国家科学技术核心竞争力的重要标志, 是国家科学技术创新发展的关键要素。

\qquad \qquad \qquad \qquad \qquad \qquad \qquad \qquad —— 国家自然科学基金 · 重大项目指南, 2014

计算科学是 21 世纪确保国家核心竞争能力的战略技术之一。

\qquad \qquad \qquad \qquad \qquad \qquad \qquad \qquad  —— 计算科学: 确保美国竞争力,2005

科学计算的核心/数学基础: 计算数学.

\subsection{矩阵计算(数值线性代数)的重要性}
If any other mathematical topic is as fundamental to the mathematical
sciences as calculus and differential equations, it is numerical linear
algebra.

\qquad \qquad \qquad \qquad \qquad \qquad \qquad\qquad—— Trefethen and Bau, 1997.

\subsubsection*{国家自然科学基金委员会关于计算数学的分类 (2018):}

• 计算数学与科学工程计算 (A0117)

- 偏微分方程数值解 (A011701)

- 流体力学中的数值计算 (A011702)

- 一般反问题的计算方法 (A011703)

- 常微分方程数值计算 (A011704)

- 数值代数 (A011705)

- 数值逼近与计算几何 (A011706)

- 谱方法及高精度数值方法 (A011707)

- 有限元和边界元方法 (A011708)

- 多重网格技术与区域分解 (A011709)

- 自适应方法 (A011720)

- 并行计算 (A011711)

\subsection{计算数学的主要任务}

• 算法设计: 构造求解各种数学问题的数值方法

• 算法分析: 收敛性、稳定性、复杂性、计算精度等

• 算法实现: 编程实现、软件开发

好的数值方法一般需要满足以下几点:

• 有可靠的理论分析, 即收敛性、稳定性等有数学理论保证

• 有良好的计算复杂性 (时间和空间)

• 易于在计算机上实现

• 要有具体的数值试验来证明是行之有效的

\subsection{矩阵计算基本问题}
数值代数: 数值线性代数 (矩阵计算) 和数值非线性代数

数值线性代数 (矩阵计算) 主要研究以下问题:

• 线性方程组求解
$$Ax=b,A \in R^{n \times n}\text{非奇异}$$

• (线性)最小二乘问题
$$
\min _{x \in \mathbb{R}^{n}}\|A x-b\|_{2} \quad, \quad A \in \mathbb{R}^{m \times n}, m \geq n
$$

•矩阵特征值问题
$$
A x=\lambda x \quad, \quad A \in \mathbb{R}^{n \times n}, \lambda \in \mathbb{C}, x \in \mathbb{C}^{n}, x \neq 0
$$

•矩阵奇异值问题
$$
A^{\top} A x=\sigma^{2} x \quad, \quad A \in \mathbb{R}^{m \times n}, \sigma \geq 0, x \in \mathbb{R}^{n}, x \neq 0
$$

•其他问题:

广义特征值问题, 二次特征值问题, 非线性特征值问题, 矩阵方程, 特征值反问题, 张量计算, . . . . . .
\begin{framed}
	
	$†$ 数值方法一般都是近似方法, 求出的解是带有误差的, 因此误差分析非常重要.
	
\end{framed}

\subsection{矩阵计算常用方法(技术或技巧)}
	• 矩阵分解
	
	• 矩阵分裂
	
	• 扰动分析

\begin{framed}
	† 问题的特殊结构对算法的设计具有非常重要的影响.
\end{framed}

\begin{framed}
	† 在编程实现时, 要充分利用现有的优秀程序库.
\end{framed}

\subsubsection*{二十世纪十大优秀算法 (SIAM News, 2000)}
1. Monte Carlo method (1946)

2. Simplex Method for Linear Programming (1947)

3. Krylov Subspace Iteration Methods (1950)

4. The Decompositional Approach to Matrix Computations (1951)

5. The Fortran Optimizing Compiler (1957)

6. QR Algorithm for Computing Eigenvalues (1959-61)

7. Quicksort Algorithm for Sorting (1962)

8. Fast Fourier Transform (1965)

9. Integer Relation Detection Algorithm (1977)

10. Fast Multipole Method (1987)

\newpage
\section{线性代数基础}
\subsection{线性空间与内积空间}
• 数域, 如: Q, R, C

• 线性空间, 如: $R^n,C^n,R^{m×n}$

• 线性相关与线性无关, 秩, 基, 维数

• 线性子空间

• 像空间 (列空间, 值域) Ran(A), 零空间 (核) Ker(A)

• 张成子空间:$span{x_1, x_2, . . . , x_k}, span(A) = Ran(A)$

\subsubsection{直和}
设$S_1, S_2$是子空间, 若$S_1 + S_2$中的任一元素都可唯一表示成

$$x = x_1 + x_2, x_1 ∈ S1, x2 ∈ S2,$$

则称$S_1 + S_2$为直和, 记为$S_1 ⊕ S_2$.

\begin{framed}
	\begin{theorem}
		
		设 $S_1$ 是 $S$ 的子空间, 则存在另一个子空间 $S_2$, 使得$$S = S_1 ⊕ S_2.$$
		
	\end{theorem}
\end{framed}

\begin{framed}
		
		{\bfseries 例:} \quad 设$A ∈\mathbb{C}^{m×n}$, 则
		$$\mathbb{C}^{n}=\operatorname{Ker}(A) \oplus \operatorname{Ran}\left(A^{*}\right), \quad \mathbb{C}^{m}=\operatorname{Ker}\left(A^{*}\right) \oplus \operatorname{Ran}(A)$$ 
		
\end{framed}

\subsubsection{内积空间}
• 内积, 内积空间, 欧氏空间, 酉空间

• 常见内积空间:

\qquad - $C^{n} : (x, y) = y∗x$

\qquad - $R^{n} : (x, y) = y^{T}x$

\qquad - $R^{m×n} : (A, B) = tr(B^{T}A)$

\subsubsection{正交与正交补}

• 正交: 向量正交, 子空间正交

• 正交补空间

\subsection{向量范数与矩阵范数}
\begin{framed}
\begin{definition}(向量范数)
	若函数$f : C^n → R $满足
	
	(1) $f(x) ≥ 0, ∀ x ∈ C^n$, 等号当且仅当 $x = 0$ 时成立;
	
	(2) $f(αx) = |α| · f(x), ∀ x ∈ C^{n}, α ∈ C;$
	
	(3) $f(x + y) ≤ f(x) + f(y), ∀x, y ∈ C^{n};$
	
	则称 f(x) 为$C^{n}$上的范数, 通常记作 $||·||$
\end{definition}
\end{framed}
相类似地,我们可以定义实数空间$R^n$上的向量范数。

常见的向量范数:

• 1-范数:$||x||_1=|x_1|+|x_2|+...+|x_n|$

• 2-范数:$||x||_2=\sqrt{|x_1|^2+|x_2|^2+...+|x_n|^2}$
 
• ∞-范数:$\|x\|_{\infty}=\max _{1 \leq i \leq n}\left|x_{i}\right|$

• p-范数:$\|x\|_{p}=\left(\sum_{i=1}^{n}\left|x_{i}\right|^{p}\right)^{1 / p}, \quad 1 \leq p<\infty$

\begin{framed}
	\begin{definition}(范数等价性)
		$\mathbb{C}^{n}$上的向量范数$||\dot||_{\alpha}$与$||\dot||_{\beta}$等价: 存在正常数$c_1, c_2,$使得
		$$c_{1}\|x\|_{\alpha} \leq\|x\|_{\beta} \leq c_{2}\|x\|_{\alpha}, \quad \forall x \in \mathbb{C}^{n}$$
	\end{definition}
\end{framed}

\begin{framed}
	\begin{theorem}
$\mathbb{C}^{n}$空间上的所有向量范数都是等价的, 特别地, 有
	$$
	\|x\|_{2} \leq\|x\|_{1} \leq \sqrt{n}\|x\|_{2}
	$$
	$$
	\|x\|_{\infty} \leq\|x\|_{2} \leq \sqrt{n}\|x\|_{\infty}
	$$
	$$
	\|x\|_{\infty} \leq\|x\|_{1} \leq n\|x\|_{\infty}
	$$
	\end{theorem}
\end{framed}

\begin{framed}
	\begin{theorem}(Cauchy-Schwartz不等式)
		设$(.,.)$是$\mathbb{C}^{n}$上的内积,则对任意$x,y \in \mathbb{C}^{n}$,有
		$$
		|(x, y)|^{2} \leq(x, x) \cdot(y, y)
		$$
	\end{theorem}
\end{framed}

\begin{framed}
	\begin{corollary}
		设$(.,.)$是$\mathbb{C}^{n}$上的内积,则$||x||\triangleq \sqrt{(x,x)}$是$\mathbb{C}^{n}$上的一个向量范数
	\end{corollary}
\end{framed}

\begin{framed}
	\begin{theorem}
		设$||.||$是$\mathbb{C}^{n}$上的一个向量范数,则$f(x)\triangleq||x||$是$\mathbb{C}^{n}$上的连续函数。
	\end{theorem}
\end{framed}

\subsubsection{矩阵范数}
\begin{framed}
	\begin{definition}(矩阵范数)
		若函数$f : \mathbb{C}^{n×n} → R$ 满足
		$(1) f(A) ≥ 0, ∀ A ∈ \mathbb{C}^{n×n},$ 等号当且仅当 $A = 0$时成立;
		$(2) f(αA) = |α| · f(A), ∀ A ∈ \mathbb{C}^{n×n}, α ∈ \mathbb{C};$
		$(3) f(A + B) ≤ f(A) + f(B), ∀A, B ∈ \mathbb{C}^{n×n};$
		则称 $f(x) $为$\mathbb{C}
		^{n×n} $上的范数, 通常记作 $||.||$。
	\end{definition}
\end{framed}

相容的矩阵范数:f(AB) ≤ f(A)f(B), ∀ A, B ∈$\mathbb{C}^{n×n}$。

\begin{framed}
	若未明确指出, 讲义所涉及矩阵范数都指相容矩阵范数
\end{framed}

\begin{framed}
	\begin{lemma}
		设 $∥ · ∥ $是 $C^{n}$上的向量范数, 则
		$$
		\|A\| \triangleq \sup _{x \in \mathbb{C}^{n}, x \neq 0} \frac{\|A x\|}{\|x\|}=\max _{\|x\|=1}\|A x\|
		$$
		是$\mathbb{C}
		^{n×n} $上的范数, 称为算子范数, 或诱导范数, 导出范数。
	\end{lemma}
\end{framed}

\begin{framed}
† 算子范数都是相容的, 且$$ \|A x\| \leq\|A\| \cdot\|x\|, \quad A \in \mathbb{C}^{n \times n}, x \in \mathbb{C}^{n}$$
\end{framed}

\begin{framed}
	† 算子范数都是相容的, 且$$ \|A x\| \leq\|A\| \cdot\|x\|, \quad A \in \mathbb{C}^{n \times n}, x \in \mathbb{C}^{n}$$
\end{framed}

\begin{framed}
† 类似地, 我们可以定义$\mathbb{C}^{m \times n}, \mathbb{R}^{n \times n}, \mathbb{R}^{m \times n}$上的矩阵范数.
\end{framed}

\begin{framed}
	\begin{lemma}
		可以证明:
		
		(1) 1-范数 (列范数):$\|A\|_{1}=\max _{1 \leq j \leq n}\left(\sum_{i=1}^{n}\left|a_{i j}\right|\right)$
		
		(2) ∞-范数 (行范数): $\|A\|_{\infty}=\max _{1 \leq i \leq n}\left(\sum_{j=1}^{n}\left|a_{i j}\right|\right)$
		
		(3) 2-范数 (谱范数):$\|A\|_{2}=\sqrt{\rho\left(A^{\top} A\right)}$
		
	\end{lemma}
\end{framed}

另一个常用范数 F-范数 $\|A\|_{F}=\sqrt{\sum_{i=1}^{n} \sum_{j=1}^{n}\left|a_{i j}\right|^{2}}$

\begin{framed}
	\begin{theorem}(矩阵范数的等价性)
		$\mathbb{R}^{n×n}$空间上的所有范数都是等价的, 特别地, 有
		$$
		\frac{1}{\sqrt{n}}\|A\|_{2} \leq\|A\|_{1} \leq \sqrt{n}\|A\|_{2},
		$$
		$$
		\frac{1}{\sqrt{n}}\|A\|_{2} \leq\|A\|_{\infty} \leq \sqrt{n}\|A\|_{2},
		$$
		$$
		\frac{1}{n}\|A\|_{\infty} \leq\|A\|_{1} \leq n\|A\|_{\infty}
,
		$$
		$$
		\frac{1}{\sqrt{n}}\|A\|_{1} \leq\|A\|_{F} \leq \sqrt{n}\|A\|_{2}
,
		$$
	\end{theorem}
\end{framed}

\subsubsection{矩阵范数的一些性质}
• 对任意的算子范数$||.||$,有$||I|| = 1$
• 对任意的相容范数$||.||$,有$||I|| \leq 1$ 
• F-范数是相容的, 但不是算子范数
• $||.||_2$和$||.||_F$酉不变范数
• $\left\|A^{\top}\right\|_{2}=\|A\|_{2},\left\|A^{\top}\right\|_{1}=\|A\|_{\infty}$
• 若$A$是正规矩阵, 则$\|A\|_{2}=\rho(A)$

\subsubsection{向量序列的收敛}
设${x^{(k)}}^{\inf}_{
k=1}$是$\mathbb{C}^n $中的一个向量序列, 如果存在 $x ∈ \mathbb{C}^n$, 使得
$$
\lim _{k \rightarrow \infty} x_{i}^{(k)}=x_{i}, \quad i=1,2, \ldots, n
$$
则称${x^{(k)}}$(按分量)收敛到 x, 记为$\lim _{k \rightarrow \infty} x^{(k)}=x$

\begin{framed}
	\begin{theorem}(矩阵范数的等价性)
		设$||.||$是$\mathbb{C}^{n}$上的任意一个向量范数, 则$\lim _{k \rightarrow \infty} x^{(k)}=x$的充要条件是
		$$
		\lim _{k \rightarrow \infty}\left\|x^{(k)}-x\right\|=0
		$$
	\end{theorem}
\end{framed}

\subsubsection{收敛速度}
设点列${εk}_{k=1}^{\inf}$收敛, 且$\lim _{k=\infty} \varepsilon_{k}=0$. 若存在一个有界常数 $0 < c < ∞$, 使得
$\lim _{k \rightarrow \infty} \frac{\left|\varepsilon_{k+1}\right|}{\left|\varepsilon_{k}\right|^{p}}=c$
则称点列 ${ε_k} $是 $p$ 次 (渐进) 收敛的. 若$ 1 < p < 2$ 或$ p = 1 $且 $c = 0,$ 则称点列是超线性收敛的.

\begin{framed}
    	$†$ 类似地, 我们可以给出矩阵序列的收敛性和判别方法.
\end{framed}

\subsection{矩阵的投影}
\subsubsection{特征值与特征向量}

• 特征多项式, 特征值, 特征向量, 左特征向量, 特征对

• n 阶矩阵 A 的谱: $σ(A) ≜ {λ1, λ2, . . . , λn}$

• 代数重数和几何重数, 特征空间

• 最小多项式

• 可对角化, 特征值分解

• 可对角化的充要条件

• 特征值估计: Bendixson 定理, 圆盘定理

\subsubsection{Bendixson定理}

设$A \in \mathbb{C}^{n \times n}$,令$H=\frac{1}{2}\left(A+A^{*}\right), S=\frac{1}{2}\left(A-A^{*}\right)$.则有
$$
\begin{array}{l}{\lambda_{\min }(H) \leq \operatorname{Re}(\lambda(A)) \leq \lambda_{\max }(H)} \\ {\lambda_{\min }(i S) \leq \operatorname{Im}(\lambda(A)) \leq \lambda_{\max }(i S)}\end{array}
$$
其中$Re(.)$和$Im(.)$分别表示实部和虚部。

\begin{framed}
	† 一个矩阵的特征值的实部的取值范围由其 Hermite 部分确定, 而虚
	部则由其 Skew-Hermite 部分确定.
\end{framed}

\subsubsection{Gerschgorin圆盘定理}
设$A=\left[a_{i j}\right] \in \mathbb{C}^{n \times n}$,定义集合
$$
\mathcal{D}_{i} \triangleq\left\{z \in \mathbb{C} :\left|z-a_{i i}\right| \leq \sum_{j=1, j \neq i}^{n}\left|a_{i j}\right|\right\}, \quad i=1,2, \ldots, n
$$
这就是A的n个Gerschgorin圆盘。

\begin{framed}
	\begin{theorem}(Gerschgorin圆盘定理)
	设$A=\left[a_{i j}\right] \in \mathbb{C}^{n \times n}$. 则 A 的所有特征值都包含在 A 的 Gerschgorin 圆盘的并集中, 即$\sigma(A) \subset \bigcup_{i=1}^{n} \mathcal{D}_{i}$
	\end{theorem}
\end{framed}


\subsubsection{投影变换与投影矩阵}
设 $S = S_1 ⊕ S_2$, 则 $S$ 中的任意向量 $x$ 都可唯一表示为
$x = x_1 + x_2, x_1 ∈ S_1, x_2 ∈ S_2.$
我们称 $x_1$ 为 $x$ 沿 $S_2$ 到 $S_1$ 上的投影, 记为 $x|_{S_1}$
.
设线性变换$ P : S → S. $如果对任意 $x ∈ S$, 都有
$P x = x|_{S_1}$
,
则称 $P$ 是从 $S$ 沿 $S_2$ 到 $S_1$ 上的 投影变换 (或 投影算子), 对应的变换矩阵
称为 投影矩阵.

\begin{framed}
	\begin{lemma}
		设 $P ∈ \mathbb{R}^{n×n} $是一个投影矩阵, 则
		\begin{equation}
		\mathbb{R}^{n}=\operatorname{Ran}(P) \oplus \operatorname{Ker}(P)
		\end{equation}
		反之, 若 $(1.3)$ 成立, 则 $P$ 是沿 $Ker(P)$ 到$ Ran(P)$ 上的投影
	\end{lemma}
\end{framed}
投影矩阵由其像空间和零空间唯一确定.

\begin{framed}
	\begin{lemma}
		若 $S_1$ 和 $S_2$ 是 $\mathbb{R}^n$的两个子空间, 且$ \mathbb{R}^n= S_1 ⊕ S_2$, 则存在唯一的
		投影矩阵 $P$, 使得
		$$
		\operatorname{Ran}(P)=\mathcal{S}_{1}, \quad \operatorname{Ker}(P)=\mathcal{S}_{2}
		$$
	\end{lemma}
\end{framed}

\subsubsection{投影矩阵的判别}
\begin{framed}
	\begin{theorem}
		矩阵 $P ∈\mathbb{R}^{n×n}$是投影矩阵的充要条件是 $P^2=P$
	\end{theorem}
\end{framed}

\subsubsection{投影算子的矩阵表示}
设 $S_1$ 和 $S_2$ 是 $\mathbb{R}^n$ 的两个 $m$ 维子空间. 如果$S_{1} \oplus \mathcal{S}_{2}^{\perp}=\mathbb{R}^{n}$ ,则存在唯一的
投影矩阵 P, 使得
$$
\operatorname{Ran}(P)=\mathcal{S}_{1}, \quad \operatorname{Ker}(P)=\mathcal{S}_{2}^{\perp}
$$
此时, 我们称 P 是 S1 上与 S2 正交的投影矩阵, 且有
$$
P=V\left(W^{\top} V\right)^{-1} W^{\top}
$$
其中 $V = [v_1, v_2, . . . , v_m] $和 $W = [w_1, w_2, . . . , w_m]$ 的列向量组分别构成
$S_1 $和 $S_2$ 的一组基.

\subsubsection{正交投影}
设 $S_1$ 是内积空间 $S $的一个子空间, $x ∈ S$, 则$x$ 可唯一分解成
$$x = x_1 + x_2, x_1 ∈ S_1, x_2 ∈ S^{⊥}_1$$
,
其中 $x_1$ 称为 $x$ 在 $S_1$ 上的正交投影.

• 若 $P$ 是沿 $S^{⊥}_1 $到 $S_1$ 上的投影变换, 则称 $P$ 为 $S_1$ 上的正交投影变换
(对应的矩阵为 正交投影矩阵), 记为 $P_{S_1}$

• 如果 $P $不是正交投影变换, 则称其为斜投影变换

\begin{framed}
	\begin{theorem}
		投影矩阵 $P ∈\mathbb{R}^{n×n}$是正交投影矩阵的充要条件 $P^{} = P$.
	\end{theorem}
\end{framed}

\begin{framed}
	\begin{theorem}
		投影矩阵 $P ∈\mathbb{R}^{n×n}$是正交投影矩阵的充要条件 $P^{\top} = P$.
	\end{theorem}
\end{framed}

\begin{framed}
	\begin{corollary}
		设 $P$ 是子空间 $S1$ 上的 正交投影变换. 令 $v_1, v_2, . . . , v_m $是$ S_1$ 的
		一组标准正交基, 则
		$$
		P=V V^{\top}
		$$
		其中$ V = [v_1, v_2, . . . , v_m].$
	\end{corollary}
\end{framed}

\begin{framed}
	\begin{property}
		设 P ∈ $\mathbb{R}^{n×n}$ 是一个正交投影矩阵, 则
		$$
		||P||_{2}=1
		$$
		且对 $∀ x ∈ \mathbb{R}^n$, 有
		$$
		||x||_{2}^{2}=||P x||_{2}^{2}+||(I-P) x||_{2}^{2}
		$$
	\end{property}
\end{framed}

\subsubsection{正交投影矩阵的一个重要应用}
\begin{framed}
	\begin{theorem}
		设 $S_1$ 是 $R_n$ 的一个子空间,$ z ∈ R_n$ 是一个向量. 则最佳逼近问题
	$$
	\min _{x \in \mathcal{S}_{1}}\|x-z\|_{2}
	$$
	的唯一解为
	 $$
	 x_{*}=P_{\mathcal{S}_{1}} z
	 $$
	 即 $S_1$ 中距离 $z$ 最近 (2-范数意义下) 的向量是 $z $在 $S
	 _1$ 上的正交投影.
	\end{theorem}
\end{framed}

\begin{framed}
	\begin{corollary}
	设矩阵 $A ∈ \mathbb{R}^{n×n}$ 对称正定, 向量 $x∗ ∈ S_1 ⊆ R_n.$ 则$ x_∗$ 是最佳逼近问题
	$$
	\min _{x \in \mathcal{S}_{1}}\|x-z\|_{A}
	$$
	的解的充要条件是
	$$
	A\left(x_{*}-z\right) \perp \mathcal{S}_{1}
	$$
	这里$\|x-z\|_{A} \triangleq\left\|A^{\frac{1}{2}}(x-z)\right\|_{2}$
	\end{corollary}
\end{framed}

\subsubsection{不变子空间}
设 $A ∈ \mathbb{R}^{n×n}$, $S$ 是 $\mathbb{R}^n$ 的一个子空间, 记
$$
A \mathcal{S} \triangleq\{A x : x \in \mathcal{S}\}
$$
\begin{framed}
	\begin{definition}
	 若$ AS ⊆ S$, 则称 $S$ 为 $A$ 的一个不变子空间.	
	\end{definition}
\end{framed}

\begin{framed}
	\begin{theorem}
	设 $x_1, x_2, . . . , x_m $是 $A$ 的一组线性无关特征向量, 则
	$$
	\operatorname{span}\left\{x_{1}, x_{2}, \ldots, x_{m}\right\}
	$$
	是 $A$ 的一个 $m $维不变子空间.
	\end{theorem}
\end{framed}


\subsubsection{不变子空间的一个重要性质}
\begin{framed}
	\begin{theorem}
		设 $A ∈ \mathbb{R}^{n×n}, X ∈ \mathbb{R}^{n×k}$ 且 $rank(X) = k$. 则 $span(X)$ 是 $A$ 的不变子空间的充要条件是存在 $B ∈ \mathbb{R}^{k×k}$ 使得
		$$AX = XB,$$
		此时, $B$ 的特征值都是 $A$ 的特征值. 
	\end{theorem}
\end{framed}

\begin{framed}
	\begin{corollary}
		设 $A ∈ \mathbb{R}^{n×n}, X ∈ \mathbb{R}^{n×k}$ 且 $rank(X) = k$. 若存在一个矩阵$B ∈ \mathbb{R}^{k×k}$ 使得 $AX = XB$, 则 $(λ, v)$ 是 $B$ 的一个特征对当且仅当$(λ, Xv)$ 是 $A$ 的一个特征对.
	\end{corollary}
\end{framed}

\subsection{矩阵标准型}
计算矩阵特征值的一个基本思想是通过相似变换, 将其转化成一个形式
尽可能简单的矩阵, 使得其特征值更易于计算. 其中两个非常有用的特殊
矩阵是 $Jordan$ 标准型和 $Schur$ 标准型.

\begin{framed}
	\begin{theorem}
	设 $A ∈ \mathbb{C}^{n×n}$ 有 $p$ 个不同特征值, 则存在非奇异矩阵$ X ∈ \mathbb{C}^{n×n}$, 使得
		$$
		X^{-1} A X=\left[\begin{array}{cccc}{J_{1}} & {} & {} & {} \\ {} & {J_{2}} & {} \\ {} & {} & {\ddots} & {} \\ {} & {} & {} & {J_{p}}\end{array}\right] \triangleq J
		$$
	其中 $J_i$ 的维数等于$ λ_i $的代数重数, 且具有下面的结构

\centering {$J_{i}=\left[\begin{array}{ccccc}{J_{i 1}} & {} & {} & {} \\ {} & {J_{i 2}} \\ {} & {} & {\ddots} & {} \\ {} & {} & {} & {J_{i \nu_{i}}}\end{array}\right]$ $J_{i k}=\left[\begin{array}{ccccc}{\lambda_{i}} & {1} & {} & {} \\ {} & {\ddots} & {\ddots} & {} \\ {} & {} & {\lambda_{i}} & {1} \\ {} & {} & {} & {\lambda_{i}}\end{array}\right]$}	

这里 $ν_i$ 为 $λ_i$ 的几何重数, $J_{ik}$ 称为 $Jordan$ 块, 每个 $Jordan$ 块对应一个特征向量	
	\end{theorem}
\end{framed}

\begin{framed}
	† Jordan 标准型在理论研究中非常有用, 但数值计算比较困难, 目前还
	没有找到十分稳定的数值算法.
\end{framed}

\begin{framed}
	\begin{corollary}
		所有可对角化矩阵组成的集合在所有矩阵组成的集合中是稠密
		的.
	\end{corollary}
\end{framed}

\newpage
\subsubsection{Schur 标准型}
\begin{framed}
	\begin{theorem}
		设 A ∈ $\mathbb{C}^{n×n}$, 则存在一个酉矩阵 $U ∈\mathbb{C}^{n×n}$ 使得	
	\centering {$U^{*} A U=\left[\begin{array}{cccc}{\lambda_{1}} & {r_{12}} & {\cdots} & {r_{1 n}} \\ {0} & {\lambda_{2}} & {\cdots} & {r_{2 n}} \\ {\vdots} & {} & {\ddots} & {\vdots} \\ {0} & {\cdots} & {0} & {\lambda_{n}}\end{array}\right] \triangleq R$ \text{或} $A=U R U^{*}$}
	
	其中 $\lambda_1$, $\lambda_2$, . . . , $\lambda_n$ 是 $A$ 的特征值 (排序任意).
	\end{theorem}
\end{framed}

关于 Schur 标准型的几点说明:

• Schur 标准型可以说是酉相似变化下的最简形式

• U 和 R 不唯一, R 的对角线元素可按任意顺序排列

• A 是正规矩阵当且仅当 定理(3.15)中的 R 是对角矩阵;

• A 是 Hermite 矩阵当且仅当 定理(3.15) 中的 R 是实对角矩阵.

\subsubsection{实 Schur 标准型}
\begin{framed}
	\begin{theorem}
		设 $A ∈ \mathbb{R}^{n×n}$, 则存在正交矩阵 $Q ∈ \mathbb{R}^{n×n}$, 使得
		$$
		Q^{\top} A Q=T
		$$
		其中 $T ∈ \mathbb{R}^{n×n}$是 拟上三角矩阵, 即$ T$ 是块上三角的, 且对角块为 $1 × 1$
		或 $2 × 2$ 的块矩阵. 若对角块是 $1 × 1$ 的, 则其就是 $A$ 的一个特征值, 若
		对角块是 $2 × 2 $的, 则其特征值是 $A$ 的一对共轭复特征值.
	\end{theorem}
\end{framed}

\subsection{几类特殊矩阵}

\subsubsection{对称正定矩阵}
设 $A ∈ \mathbb{C}^{n×n}$.

A 是 半正定 ⇐⇒ $\operatorname{Re}\left(x^{*} A x\right) \geq 0, \forall x \in \mathbb{C}^{n}$

A 是 正定 ⇐⇒$\operatorname{Re}\left(x^{*} A x\right)>0, \forall x \in \mathbb{C}^{n}, x \neq 0$

A 是 Hermite 半正定 ⇐⇒ A Hermite 且半正定

A 是 Hermite 正定 ⇐⇒ A Hermite 且正定

\begin{framed}
	† 正定和半正定矩阵不要求是对称或 Hermite 的
\end{framed}

\begin{framed}
	\begin{theorem}
		设 $A ∈ \mathbb{C}^{n×n}$. 则 $A$ 正定 (半正定) 的充要条件是矩阵$ H =\frac{1}{2}(A +A∗)$ 正定 (半正定). 
	\end{theorem}
\end{framed}

\begin{framed}
	\begin{theorem}
		设 $A ∈ \mathbb{R}^{n×n}$. 则 $A$ 正定 (或半正定) 的充要条件是对任意非零向量 $x ∈ \mathbb{R}^n$ 有 $x^{⊺}Ax > 0 $(或$ x^{⊺}Ax ≥ 0$). 
	\end{theorem}
\end{framed}


\subsubsection{矩阵平方根}
\begin{framed}
	\begin{theorem}
	 设 $A ∈ \mathbb{C}^{n×n}$ 是 Hermite 半正定, $k $是正整数. 则存在唯一的Hermite 半正定矩阵 $B ∈ \mathbb{C}^{n×n}$ 使得
	 $$B^k = A.$$
	 同时, 我们还有下面的性质:
	 (1) $BA = AB$, 且存在一个多项式 $p(t)$ 使得$ B = p(A)$;
	 (2) $rank(B) = rank(A)$, 因此, 若$ A$ 是正定的, 则 $B$ 也正定;
	 (3) 如果 $A$ 是实矩阵的, 则 $B$ 也是实矩阵.
	\end{theorem}
\end{framed}

特别地, 当 k = 2 时, 称 B 为 A 的平方根, 通常记为 $A^{\frac{1}{2})}$.

Hermite 正定矩阵与内积之间有下面的关系
\begin{framed}
	\begin{theorem}
		设 $(·, ·)$ 是 $\mathbb{C}^n$上的一个内积, 则存在一个 Hermite 正定矩阵 $A  \mathbb{C}^{n×n}$使得
		$$(x, y) = y^∗Ax.$$
		反之, 若 $A ∈  \mathbb{C}^{n×n} $是 Hermite 正定矩阵, 则
		$$
		f(x, y) \triangleq y^{*} A x
		$$
		是 $\mathbb{C}^n $上的一个内积. 
	\end{theorem}
\end{framed}

\begin{framed}
	† 上述性质在实数域中也成立.
\end{framed}


\subsubsection{对角占优矩阵}

\begin{framed}
	\begin{definition}
	设$ A ∈\mathbb{C}^{n×n}$, 若
	$$
	\left|a_{i i}\right| \geq \sum_{j \neq i}\left|a_{i j}\right|
	$$
	对所有 $i = 1, 2, . . . , n $都成立, 且至少有一个不等式严格成立, 则称 A
	为弱行对角占优. 若对所有 $i = 1, 2, . . . , n$ 不等式都严格成立, 则称 A
	是严格行对角占优. 通常简称为弱对角占优和严格对角占优.	
	\end{definition}
\end{framed}

\begin{framed}
	† 类似地, 可以定义弱列对角占优 和 严格列对角占优.
\end{framed}

\subsubsection{可约与不可约}
设$A \in \mathbb{R}^{n \times n}$, 若存在置换矩阵 P, 使得$P A P^{\top}$为块上三角, 即
$$
P A P^{\top}=\left[\begin{array}{cc}{A_{11}} & {A_{12}} \\ {0} & {A_{22}}\end{array}\right]
$$
其中 $A_{11} \in \mathbb{R}^{k \times k}(1 \leq k<n)$, 则称 A 为可约, 否则不可约.


\begin{framed}
	\begin{theorem}
		设 $A \in \mathbb{R}^{n \times n}$, 指标集$\mathbb{Z}_{n}=\{1,2, \ldots, \ldots, n\}$. 则 A 可约的充要条
		件是存在非空指标集 $J \subset \mathbb{Z}_{n}$ 且 $J \neq \mathbb{Z}_{n}$, 使得
		$$
		a_{i j}=0, \quad i \in J \text{且} j \in \mathbb{Z}_{n} \backslash J
		$$
		这里 $\mathbb{Z}_n \ J $表示 J 在$ \mathbb{Z}_n $中的补集.	
	\end{theorem}
\end{framed}

\begin{framed}
	\begin{theorem}
		若$A \in \mathbb{C}^{n \times n}$严格对角占优, 则 A 非奇异	
	\end{theorem}
\end{framed}

\begin{framed}
	\begin{theorem}
		若$A \in \mathbb{C}^{n \times n}$不可约对角占优, 则 A 非奇异	
	\end{theorem}
\end{framed}

\subsubsection{其他常见特殊矩阵}
• 带状矩阵:
$a_{i j} \neq 0$ only if $-b_{u} \leq i-j \leq b_{l}$
, 其中 $b_u$ 和 $b_l$ 为非负整数, 分别称为下
带宽和上带宽, $b_u + b_l + 1 $称为 A 的带宽

• 上 Hessenberg 矩阵: $a_{ij} = 0\quad for\quad i − j > 1$,
$$
\left[\begin{array}{ccccc}{*} & {*} & {*} & {\dots} & {*} \\ {*} & {*} & {*} & {\dots} & {*} \\ { } & {*} & {*} & {\dots} & {*} \\ { } & { } & {\ddots} & {\ddots} & {\vdots}\\ { } & { } & { } & {*} & {*}\end{array}\right]
$$

• 下 Hessenberg 矩阵

• Toeplitz 矩阵
$$
T=\left[\begin{array}{cccc}{t_{0}} & {t_{-1}} & {\dots} & {t_{-n+1}} \\ {t_{1}} & {\ddots} & {\ddots} & {\vdots} \\ {\vdots} & {\ddots} & {\ddots} & {t_{-1}} \\ {t_{n-1}} & {\cdots} & {t_{1}} & {t_{0}}\end{array}\right]
$$

• 循环矩阵 (circulant):
$$
C=\left[\begin{array}{ccccc}{c_{0}} & {c_{n-1}} & {c_{n-2}} & {\cdots} & {c_{1}} \\ {c_{1}} & {c_{0}} & {c_{n-1}} & {\cdots} & {c_{2}} \\ {c_{2}} & {c_{1}} & {c_{0}} & {\cdots} & {c_{3}} \\ {\vdots} & {\vdots} & {\vdots} & {\ddots} & {\vdots} \\ {c_{n-1}} & {c_{n-2}} & {c_{n-3}} & {\cdots} & {c_{0}}\end{array}\right]
$$

• Hankel 矩阵:
$$
H=\left[\begin{array}{ccccc}{h_0} & {h_1} & {\cdots} & {h_{n-2}} & {h_{n-1}} \\ {h_1} & {\ddots} & {\ddots} & {\ddots} & {h_n}\\ {\vdots} & {\ddots} & {\ddots} & {\ddots} & {\vdots}  \\ {h_{n-2}} & {\cdots} & {\cdots} & {\cdots} & {h_{2n-2}} \\ {h_{n-1}} & {h_n} & {\dots} & {h_{2n-2}} & {h_{2n-1}}\end{array}\right]
$$

\subsection{Kronecker 积}
\begin{framed}
	\begin{definition}
		设$A \in \mathbb{C}^{m \times n}, B \in \mathbb{C}^{p \times q}$, 则 A 与 B 的 Kronecker 积定义为
			$$
			A \otimes B=\left[\begin{array}{cccc}{a_{11} B} & {a_{12} B} & {\cdots} & {a_{1 n} B} \\ {a_{21} B} & {a_{22} B} & {\cdots} & {a_{2 n} B} \\ {\vdots} & {\vdots} & {\ddots} & {\vdots} \\ {a_{m 1} B} & {a_{m 2} B} & {\cdots} & {a_{m n} B}\end{array}\right] \in \mathbb{C}^{m p \times n q}
			$$
			
			Kronecker 积也称为直积, 或张量积.
	\end{definition}
\end{framed}

\begin{framed}
	† 任意两个矩阵都存在 Kronecker 积, 且 $A ⊗ B$ 和 $B ⊗ A$ 是同阶矩阵,
	但通常 $A \otimes B \neq B \otimes A$
\end{framed}


\subsubsection{基本性质}
\begin{framed}
	(1) $(αA) ⊗ B = A ⊗ (αB) = α(A ⊗ B), ∀ α ∈ \mathbb{C}$
	
	(2) $(A ⊗ B)^{⊺} = A^{⊺} ⊗ B^{⊺}, (A ⊗ B)^{∗} = A^{∗} ⊗ B^{∗}$
	
	(3) $(A ⊗ B) ⊗ C = A ⊗ (B ⊗ C)$
	
	(4) $(A + B) ⊗ C = A ⊗ C + B ⊗ C$
	
	(5) $A ⊗ (B + C) = A ⊗ B + A ⊗ C$
	
	(6)混合积:$ (A ⊗ B)(C ⊗ D) = (AC) ⊗ (BD)$
	
	(7) $(A_1 ⊗ A_2 ⊗ · · · ⊗ A_k)(B_1 ⊗ B_2 ⊗ · · · ⊗ B_k)
	= (A_1B_1) ⊗ (A_2B_2) ⊗ · · · ⊗ (A_kB_k)$
	
	(8) $(A_1 ⊗ B_1)(A_2 ⊗ B_2)· · ·(A_k ⊗ B_k)
	= (A_1A_2 · · · A_k) ⊗ (B_1B_2 · · · B_k)$
	
	(9) $rank(A ⊗ B) = rank(A)rank(B)$
	
\end{framed}

\begin{framed}
	\begin{theorem}
		设 $A ∈ \mathbb{C}^{m×m}, B ∈ \mathbb{C}^{n×n},$ 并设 $(λ, x)$ 和 $(µ, y)$ 分别是$ A $和$ B $的一个特征对, 则 $(λµ, x ⊗ y) $是 $A ⊗ B$ 的一个特征对. 由此可知,$ B ⊗ A $与$A ⊗ B$ 具有相同的特征值.	
	\end{theorem}
\end{framed}

\begin{framed}
	\begin{theorem}
		设 $A ∈ \mathbb{C}^{m×m}$, $B ∈ \mathbb{C}^{n×n}$, 则
		
		(1) $tr(A ⊗ B) = tr(A)tr(B)$ ;
		
		(2) $det(A ⊗ B) = det(A)^n det(B)^m $;
		
		(3) $A ⊗ I_n + I_m ⊗ B $的特征值为 $λ_i + µ_j$ , 其中 $λ_i$ 和 $µ_j$ 分别为 A 和 B 的特征值;
		
		(4) 若 A 和 B 都非奇异, 则 $(A ⊗ B)−1 = A^{−1} ⊗ B^{−1}$;	
	\end{theorem}
\end{framed}

\begin{framed}
	\begin{corollary}
		设$A=Q_{1} \Lambda_{1} Q_{1}^{-1}, B=Q_{2} \Lambda_{2} Q_{2}^{-1}$,则
		$$
		A \otimes B=\left(Q_{1} \otimes Q_{2}\right)\left(\Lambda_{1} \otimes \Lambda_{2}\right)\left(Q_{1} \otimes Q_{2}\right)^{-1}
		$$	
	\end{corollary}
\end{framed}

\begin{framed}
	\begin{theorem}
		设$A \in \mathbb{C}^{m \times m}, B \in \mathbb{C}^{n \times n}$,则存在 m + n 阶置换矩阵 P 使得
		
		$$
		P^{\top}(A \otimes B) P=B \otimes A
		$$
		
	\end{theorem}
\end{framed}

\begin{framed}
	\begin{theorem}
		设矩阵 $X = [x1, x2, . . . , xn] ∈ \mathbb{R}^{
		m×n}$, 记 $vec(X)$为 $X$ 按列拉成的
		mn 维列向量, 即
		$$
		\operatorname{vec}(X)=\left[x_{1}^{\top}, x_{2}^{\top}, \ldots, x_{N}^{\top}\right]^{\top}
		$$
		则有
		$$vec(AX) = (I ⊗ A)vec(X), vec(XB) = (B^{\top}⊗ I)vec(X),$$
		以及
		$$(A ⊗ B)vec(X) = vec(BXA^{\top}
		)$$
	\end{theorem}
\end{framed}

\newpage
\section{线性方程组直接解法}

{\bfseries 线性方程组的求解方法}

• 直接法: $LU$ 分解, $Cholesky$ 分解, ...

• 迭代法: 古典迭代法, $Krylov$ 子空间迭代法

本章介绍直接法, 即 Gauss 消去法 或 PLU 分解

直接法优点: 稳定可靠 → 在工程界很受欢迎

直接法缺点: 运算量大 $O(n^3)$ → 不适合大规模稀疏线性方程组(针对特殊结构矩阵的快速方法除外)

\subsection{Gauss 消去法和 LU 分解}

4.1.1 $LU$ 分解

4.1.2 $LU$ 分解的实现

4.1.3 $IKJ$ 型 $LU$ 分解

4.1.4 待定系数法计算 $LU$ 分解

4.1.5 三角方程求解

4.1.6 选主元 $LU$ 分解

4.1.7 矩阵求逆

\subsubsection{LU 分解}
考虑线性方程组
\begin{equation}
A x=b
\end{equation}
其中 $A ∈ \mathbb{R}^{n×n}$ 非奇异, $b ∈ \mathbb{R}^n$ 为给定的右端项.

\begin{framed}
	Gauss 消去法本质上就是对系数矩阵 A 进行 LU 分解:
	\begin{equation}
	A=L U
	\end{equation}
	其中 L 是单位下三角矩阵, U 为非奇异上三角矩阵.
\end{framed}

分解()就称为$LU$分解

\begin{equation}
A x=b \Longleftrightarrow\left\{\begin{array}{l}{L y=b} \\ {U x=y}\end{array} | \Longrightarrow \text{ 只需求解两个三角方程组} \right.
\end{equation}

\begin{table}  
	%\caption{设置表格总长}  
	\begin{tabular*}{16cm}{ll}  
		\hline  
		算法4.1: & Gauss 消去法 \\  
		\hline  
		1:   &将 $A$ 进行 $LU$ 分解:\\  
		2:   &$A = LU$, 其中 $L$ 为单位下三角矩阵, $U$ 为非奇异上三角矩阵;\\
		3:   &向前回代: 求解 $Ly = b$, 即得$ y = L^{-1}b$\\
		4:   &向后回代: 求解 $Ux = y$, 即得 $x = U^{-1}y = (LU)^{-1}b = A^{-1}b$.\\  
		\hline  
	\end{tabular*}  
\end{table} 

 \begin{framed}
 	$†$ 需要指出的是: $A$ 非奇异, 则解存在唯一, 但并不一定存在 $LU$ 分解!
 \end{framed}

\begin{framed}
	\begin{theorem}(LU 分解的存在性和唯一性)
	设 $A ∈ \mathbb{R}^{n×n}$. 则存在唯一的单位下
	三角矩阵 $L$ 和非奇异上三角矩阵 $U$, 使得 $A = LU$ 的充要条件是 $A$ 的
	所有顺序主子矩阵 $A_k = A(1:k, 1:k)$ 都非奇异, $k = 1, 2, . . . , n.$	
	\end{theorem}
\end{framed}

\newpage
\subsubsection{LU 分解的实现 — 矩阵初等变换}
给定一个矩阵
\begin{equation*}
A=\left[\begin{array}{cccc}{a_{11}} & {a_{12}} & {\cdots} & {a_{1 n}} \\ {a_{21}} & {a_{22}} & {\cdots} & {a_{2 n}} \\ {\vdots} & {} & {\ddots} & {} \\ {a_{n 1}} & {a_{n 2}} & {\cdots} & {a_{n n}}\end{array}\right] \in \mathbb{R}^{n \times n}
\end{equation*}

• 第一步: 假定 $a_{11} \neq  0$, 构造矩阵

\centering {$L_{1}=\left[\begin{array}{ccccc}{1} & {0} & {0} & {\cdots} & {0} \\ {l_{21}} & {1} & {0} & {\cdots} & {0} \\ {l_{31}} & {0} & {1} & {\cdots} & {0} \\ {\vdots} & {} & {} & {\ddots} \\ {l_{n 1}} & {0} & {0} & {\cdots} & {1}\end{array}\right]$ $\text{,其中}$ $l_{i 1}=\frac{a_{i 1}}{a_{11}}, i=2,3, \dots, n $}
\\
易知 $L_1$ 的逆为
$$
L_{1}^{-1}=\left[\begin{array}{ccccc}{1} & {0} & {0} & {\cdots} & {0} \\ {-l_{21}} & {1} & {0} & {\cdots} & {0} \\ {-l_{31}} & {0} & {1} & {\cdots} & {0} \\ {\vdots} & {} & {\ddots} & {} \\ {-l_{n 1}} & {0} & {0} & {\cdots} & {1}\end{array}\right]
$$
\\
用 $L^{−1}_1 $左乘 $A$, 并将所得到的矩阵记为 $A(1)$
, 则
$$
A^{(1)}=L_{1}^{-1} A\left[\begin{array}{cccc}{a_{11}} & {a_{12}} & {\cdots} & {a_{1 n}} \\ {0} & {a_{22}^{(1)}} & {\cdots} & {a_{2 n}^{(1)}} \\ {\vdots} & {\vdots} & {\ddots} & {} \\ {0} & {a_{n 2}^{(1)}} & {\cdots} & {a_{n n}^{(1)}}\end{array}\right]
$$
\\
即左乘 $L^{−1}_1 $后, $A$ 的第一列中除第一个元素外其它都变为$ 0$.
\\
• 第二步: 将上面的操作作用在 $A^{(1)}$ 的子矩阵 $A^{(1)}(2 : n, 2 : n)$ 上, 将
其第一列除第一个元素外都变为 $0:$ 假定$ a^{(1)}_{22} \neq 0$, 构造矩阵

\centering {$L_{2}=\left[\begin{array}{ccccc}{1} & {0} & {0} & {\cdots} & {0} \\ {0} & {1} & {0} & {\cdots} & {0} \\ {0} & {l_{32}} & {1} & {\cdots} & {0} \\ {\vdots} & {\vdots} & {\ddots} & {\ddots} \\ {0} & {l_{n 2}} & {0} & {\cdots} & {1}\end{array}\right]$ $\text{,其中}$ $l_{i 2}=\frac{a_{i 2}^{(1)}}{a_{22}^{(1)}}, i=3,4, \ldots, n$}

用$L_{2}^{-1}$左乘$A^{(1)}$
, 并将所得到的矩阵记为$A^{(2)}$
, 则
$$
A^{(2)}=L_{2}^{-1} A=L_{2}^{-1} L_{1}^{-1}A=\left[\begin{array}{ccccc}{a_{11}} & {a_{12}} & {a_{13}} & {\dots} &{a_{1 n}} \\ {0} & {a_{22}^{(1)}} & {a_{23}^{(1)}} & {\dots} & {a_{2 n}^{(1)}} \\ {0} & {0} & {a_{33}^{(2)}} & {\dots} & {a_{3 n}^{(2)}} \\ {\vdots} & {\vdots} & {\vdots} & {\ddots} &{ } \\ {0} & {0} & {a_{n 3}^{(2)}} & {\cdots} & {a_{n n}^{(2)}}\end{array}\right]
$$

依此类推, 假定$a_{k k}^{(k-1)} \neq 0(k=3,4, \dots, n-1)$, 则我们可以构造一系
列的矩阵$ L_3, L_4, . . . , L_{n−1}$, 使得
$$
L_{n-1}^{-1} \cdots L_{2}^{-1} L_{1}^{-1} A=\left[\begin{array}{ccccc}{a_{11}} & {a_{12}} & {a_{13}} & {\dots} & {a_{1 n}} \\ {0} & {a_{22}^{(1)}} & {a_{23}^{(1)}} & {\dots} & {a_{2 n}^{(1)}} \\ {0} & {0} & {a_{33}^{(2)}} & {\cdots} & {a_{3 n}^{(2)}} \\ {\vdots} & {\vdots} & {\vdots} & {\ddots} \\ {0} & {0} & {0} & {\cdots} & {a_{n n}^{(n-1)}}\end{array}\right] \triangleq U \rightarrow \text{上三角}
$$

于是可得 $A = LU$ 其中
$$
L=L_{1} L_{2} \cdots L_{n-1}=\left[\begin{array}{ccccc}{1} & {0} & {0} & {\cdots} &{0}\\ {l_{21}} & {1} & {0} & {\cdots} &{0}\\ {l_{31}} & {l_{32}} & {1} & {\cdots} &{0}\\ {\vdots} & {\vdots} & { }& {\ddots} & { }\\ {l_{n 1}} & {l_{n 2}} & {l_{n 3}} & {\cdots} &{1}\end{array}\right]
$$

\begin{table}  
	%\caption{设置表格总长}  
	\begin{tabular*}{16cm}{ll}  
		\hline  
		算法4.2: &LU 分解 \\  
		\hline  
		1:   &for k = 1 to n − 1 do\\  
		2:   &\qquad for i = k + 1 to n do\\
		3:   &\qquad \qquad $l_{ik} = a_{ik}/a_{kk}$ \% 计算 L 的第 k 列\\
		4:   &\qquad end for\\
		5:   &\qquad for j = k to n do\\
		6:   &\qquad \qquad $u_{kj} = a_{kj}$ \% 计算 U 的第 k 行\\
		7:   &\qquad end for\\
		8:   &\qquad end for i = k + 1 to n do\\
		9:   & \qquad \qquad for j = k + 1 to n do\\
		10:   &\qquad \qquad \qquad $a_{ij}= a_{ij} − l_{ik}u_{kj}$ \% 更新 A(k + 1 : n, k + 1 : n)\\
		11:   &\qquad \qquad end for\\
		12:   &\qquad end for\\
		13:   &end for\\
		\hline  
	\end{tabular*}  
\end{table} 

{\bfseries Gauss 消去法的运算量}

由算法 4.2 可知, LU 分解的运算量 (加减乘除) 为
$$
\sum_{i=1}^{n-1}\left(\sum_{j=i+1}^{n} 1+\sum_{j=i+1}^{n} \sum_{k=i+1}^{n} 2\right)=\sum_{i=1}^{n-1}\left(n-i+2(n-i)^{2}\right)=\frac{2}{3} n^{3}+O\left(n^{2}\right)
$$
加上回代过程的运算量$O\left(n^{2}\right)$, 总运算量为$\frac{2}{3} n^{3}+O\left(n^{2}\right)$

\begin{framed}
	† 评价算法的一个主要指标是执行时间, 但这依赖于计算机硬件和编
	程技巧等, 因此直接给出算法执行时间是不太现实的. 所以我们通常
	是统计算法中算术运算 (加减乘除) 的次数.
\end{framed}

\begin{framed}
	† 在数值算法中, 大多仅仅涉及加减乘除和开方运算. 一般地, 加减运
	算次数与乘法运算次数具有相同的量级, 而除法运算和开方运算次
	数具有更低的量级.
\end{framed}

\begin{framed}
	† 为了尽可能地减少运算量, 在实际计算中, 数, 向量和矩阵做乘法运
	算时的先后执行次序为: 先计算数与向量的乘法, 然后计算矩阵与向
	量的乘法, 最后才计算矩阵与矩阵的乘法.
\end{framed}

{\bfseries 矩阵 L 和 U 的存储}

当 A 的第 i 列被用于计算 L 的第 i 列后, 在后面的计算中不再被使用.

同样地, A 的第 i 行被用于计算 U 的第 i 行后, 在后面计算中也不再使用

\begin{framed}
	为了节省存储空间, 在计算过程中将 L 的第 i 列存放在 A 的第 i 列, 将
	U 的第 i 行存放在 A 的第 i 行, 这样就不需要另外分配空间存储 L 和 U.
\end{framed}

计算结束后, A 的上三角部分为 U, 其绝对下三角部分为 L 的绝对下三角
部分 (L 的对角线全部为 1, 不需要存储).

\begin{table}  
	%\caption{设置表格总长}  
	\begin{tabular*}{16cm}{ll}  
		\hline  
		算法4.3: & LU 分解 \\  
		\hline  
		1:   &for k = 1 to n − 1 do\\  
		2:   &\qquad for i = k + 1 to n do\\
		3:   &\qquad \qquad $a_{ik} = a_{ik}/a_{kk}$\\
		4:   &\qquad \qquad for j = k + 1 to n do\\
		5:    &\qquad \qquad \qquad $a_{ij} = a_{ij} − a_{ik}a_{kj}$\\
		6:    &\qquad \qquad end for\\
		7:    &\qquad end for\\
		8:    &end for \\
		\hline  
	\end{tabular*}  
\end{table} 

%%%%%%%%%%%%%%%%%%%%%%%%%%%%%%%%%%%%%%%%%%%%%%%%
\begin{framed}
	
	† 根据指标的循环次序, 算法 $4.3$ 也称为 $KIJ$ 型 $LU$ 分解. 实际计算中一般不建议使用: 对指标$k$ 的每次循环, 都需要更新 $A$ 的第 $k + 1 $至第$n $行, 这种反复读取数据的做法会使得计算效率大大降低.对于按行存储的数据结构, 一般采用后面介绍的 $IKJ$ 型 $LU$分解.
	
\end{framed}






	
\begin{lstlisting}[language={[ANSI]C}] 
% Matlab code 1 : LU 分解
function A = mylu(A)
n=size(A,1);
	for k=1:n-1
		if A(k,k) == 0
			fprintf('Error: A(%d,%d)=0!\n', k, k);
		return;
	end
	for i=k+1:n
		A(i,k)=A(i,k)/A(k,k);
		for j=k+1:n
			A(i,j)=A(i,j)-A(i,k)*A(k,j);
		end
	end
end
\end{lstlisting}

为了充分利用 Matlab 的向量运算优势, 提高运算效率, 程序可改写为
\begin{lstlisting}[language={[ANSI]C}] 
% Matlab code 2 : LU 分解
function A = mylu(A)
n=size(A,1);
for k=1:n-1
	if A(k,k) == 0
		fprintf('Error: A(%d,%d)=0!\n', k, k);
		return;
	end
	A(k+1:n,k)=A(k+1:n,k)/A(k,k);
	A(k+1:n,k+1:n)=A(k+1:n,k+1:n)-A(k+1:n,k)*A(k,k+1:n);
end
\end{lstlisting}

\maketitle
\newpage

\subsubsection{IKJ 型 LU 分解}

如果数据是按行存储的, 如 C/C++, 我们一般采用下面的 IKJ 型 LU 分解.

\begin{table}  
	%\caption{设置表格总长}  
	\begin{tabular*}{16cm}{ll}  
		\hline  
		算法4.4: & LU 分解 \\  
		\hline  
		1:   &for i = 2 to n do\\  
		2:   &\qquad for k = 1 to i − 1 do\\
		3:   &\qquad \qquad $a_{ik} = a_{ik}/a_{kk}$\\
		4:   &\qquad \qquad for j = k + 1 to n do\\
		5:    &\qquad \qquad \qquad $a_{ij} = a_{ij} − a_{ik}a_{kj}$\\
		6:    &\qquad \qquad end for\\
		7:    &\qquad end for\\
		8:    &end for \\
		\hline  
	\end{tabular*}  
\end{table} 















\end{document}
