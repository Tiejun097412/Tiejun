% !Mode:: "TeX:UTF-8"
\documentclass[12pt,a4paper]{article}
\input{en_preamble.tex}
\input{xecjk_preamble.tex}

\title{学习报告}
\author{作者:周铁军}
\date{\chntoday}
\begin{document}
\maketitle
\newpage


\section{均匀B-Spline曲线}
\begin{figure}[ht]
	\centering
	\includegraphics[scale=0.6]{./figures/1.png}
	\caption{}
	\label{fig:label}	
\end{figure}
\begin{figure}[ht]
	\centering
	\includegraphics[scale=0.6]{./figures/2.png}
	\caption{}
	\label{fig:label}	
\end{figure}

\subsection{问题一}
我们注意到图1和图2的区别在于,(1,1)与(2,2)之间多了一条连线。

经过多番考虑和实验分析,应该只是我连接控制点时多画了而已。

首先,三次均匀B样条曲线的矩阵形式表达如下:

$$
\mathrm{p}(\mathrm{u})=\left(\begin{array}{cccc}{\mathrm{d}_{0}} & {\mathrm{d}_{1}} & {\ldots} & {\mathrm{d}_{\mathrm{n}}}\end{array}\right)\left(\begin{array}{c}{\mathrm{N}_{0, \mathrm{3}}(\mathrm{u})} \\ {\mathrm{N}_{1, \mathrm{3}}(\mathrm{u})} \\ {\vdots} \\ {\mathrm{N}_{\mathrm{n}, \mathrm{3}}(\mathrm{u})}\end{array}\right)
$$

只需要确定控制顶点$d_i$、曲线的次数k以及基函数$N_{i,k}(u)$,就完全确定了曲线。

所以,B样条曲线是受控制点的影响,关于图中控制点的连线,只是为了可视化更清楚而已。

下面是不对控制点连线的图像:
\begin{figure}[ht]
	\centering
	\includegraphics[scale=0.4]{./figures/noline.png}
	\caption{不对控制点连线图}
	\label{fig:label}	
\end{figure}
\begin{figure}[ht]
	\centering
	\includegraphics[scale=0.4]{./figures/nolineb.png}
	\caption{不对控制点连线放大图}
	\label{fig:label}	
\end{figure}

\newpage
对于我这么问题,首先内外两个控制点围成的矩形区域(后面简称矩形区域),不妨暂且先看做两组控制点$P=[d_1,d_2,...,d_p]$、$Q=[d'_1,d'_2,...,d'_q]$,

因为我要求的B样条曲线应该是两组控制点共同约束得到的曲线,
曲线表达式应该为:
$$
\mathrm{p}(\mathrm{u})=\left(\begin{array}{cc}{P} & {Q}\end{array}\right)\left(\begin{array}{c}{\mathrm{N}_{0, \mathrm{3}}(\mathrm{u})} \\ {\mathrm{N}_{1, \mathrm{3}}(\mathrm{u})} \\ {\vdots} \\ {\mathrm{N}_{\mathrm{n}, \mathrm{3}}(\mathrm{u})}\end{array}\right)
$$

故我们计算时,应该是把控制点看成一组点。


相反,如果分成两组点计算,则我们得到的结果是两个矩形区域分别约束得到的B-Spline曲线,只不过同框罢了。

和单独计算两次矩形约束区域无异。

{\bfseries 以上是我对于求均匀B-Spline曲线时遇到的问题的分析及依据,如果有考虑不充分的地方还请老师雅正。}
     
     

\end{document}
