% !Mode:: "TeX:UTF-8"
\documentclass[12pt,a4paper]{article}
\input{en_preamble.tex}
\input{xecjk_preamble.tex}

\title{学习报告}
\author{作者:周铁军}
\date{\chntoday}
\begin{document}
\maketitle
\newpage


\section{均匀B-Spline曲线}
\subsection{问题重现}

控制顶点:$d_0,d_1,d_2,d_3,...,d_m$,这里为了闭合,我是使$d_m=d_0$

诚然,对于控制顶点而言,确实实现了闭合。

我使用的算法一是:
\begin{equation}
\boldsymbol{p}(u)=\sum_{j=0}^{n} \boldsymbol{d}_{j} \mathbf{N}_{j, k}(u)=\sum_{j=i-k}^{i} \boldsymbol{d}_{j} \mathbf{N}_{j, k}(u)
\end{equation}
其中$d_i$为控制顶点,$N_{ij}(u)$为基函数。

基函数计算采用De Boor-Cox递推公式。
\begin{figure}[ht]
	\centering
	\includegraphics[scale=0.6]{./figures/D.png}
	\caption{De Boor - Cox递推公式}
	\label{fig:label}	
\end{figure}




得到的结果如下:
\begin{figure}[ht]
	\centering
	\includegraphics[scale=0.4]{./figures/false1.png}
	\caption{算法一}
	\label{fig:label}	
\end{figure}

为了验证算法结果,我采用了另一种算法(只是为了检验算法一,故只验证了大矩形),算法如下:

先对参数变换:
\begin{equation}
u=u(t)=(1-t) u_{i}+t u_{i+1}, \quad t \in[0,1] ; i=k, k+1, \cdots, n
\end{equation}

则B样条曲线方程改写为:
\begin{equation}
s_{i}(t)=p(u(t))=\sum_{j=i=k}^{i} d_{j} \mathrm{N}_{j, k}(u(t)), \quad t \in[0,1] ; i=k, k+1, \cdots, n
\end{equation}
改成矩阵形式:
\begin{equation}
s_{i}(t)=\left[\begin{array}{lllll}{1} & {t} & {t^{2}} & {\cdots} & {t^{k}}\end{array}\right] \boldsymbol{M}_{k}\left[\begin{array}{c}{\boldsymbol{d}_{i-k}} \\ {\boldsymbol{d}_{i-k+1}} \\ {\vdots} \\ {\boldsymbol{d}_{i}}\end{array}\right], \quad t \in[0,1] ; i=k, k+1, \cdots, n
\end{equation}

这里矩阵$M_k$用的是书上给出的,验证了计算了部分值,没有出错,网上查找资料时,亦为该矩阵。
\begin{equation}
M_{3}=\frac{1}{6}\left[\begin{array}{cccc}{1} & {4} & {1} & {0} \\ {-3} & {0} & {3} & {0} \\ {3} & {-6} & {3} & {0} \\ {-1} & {3} & {-3} & {1}\end{array}\right]
\end{equation}
得到结果如下:
\begin{figure}[ht]
	\centering
	\includegraphics[scale=0.4]{./figures/false2.png}
	\caption{算法二}
	\label{fig:label}	
\end{figure}

算法二的结果与算法一如出一折,在左下有个缺口,{\bfseries 不过却也验证算法一的过程应该是没有出错。}


{\bfseries \red{最后,在书上注意到如下性质:}}

\subsection{k次B样条闭曲线与开曲线的统一表示问题}
\begin{figure}[ht]
	\centering
	\includegraphics[scale=0.4]{./figures/1.png}
	\caption{与开曲线表示不统一的B样条曲线(m=6,k=3)}
	\label{fig:label}	
\end{figure}

给定控制顶点$d_j(j=0,1,...,m)$其中$d_0 = d_m$

关于统一表示,可按如下步骤确定:

(1)确定控制顶点下标的上界值$n-m+k-r$,其中$r$为重复度

(2)按$d_{m+j}-d_j$$(j=0,1,...,k-r)$决定$k-r+1$个重顶点:$d_m=d_0,d_{m+1}=d_1,...,d_{m+k-r}=d_{k-r}$.

(3)定义该闭曲线的控制顶点$d_j(j=0,1,..,m,m+1,...,n)$.

即顶点下标的上界从给定的$m$增大到$n$,增加了$k-r$个重顶点,加上原来已有的一个重顶点,共有$k-r+1$个重顶点。按如上确定后,节点矢量与曲线定义域就与开曲线完全相同。


\subsection{问题解决}
从书中得知闭合的均匀B样条曲线的重复度$r=1$,则重顶点数应该为$k-r+1=3$个。

即对于给定控制顶点$d_j(j=0,1,...,m)$其中已经有$d_m=d_0$,我们计算时需要增加$d_{m+1}=d_1,d_{m+2}=d_{2}$来计算。

增加重顶点后,得到算法一图像:
\begin{figure}[ht]
	\centering
	\includegraphics[scale=0.4]{./figures/right1.png}
	\caption{算法一图像}
	\label{fig:label}	
\end{figure} 

得到算法二图像:    
\begin{figure}[ht]
	\centering
	\includegraphics[scale=0.4]{./figures/right2.png}
	\caption{算法二图像}
	\label{fig:label}	
\end{figure}   

\end{document}
