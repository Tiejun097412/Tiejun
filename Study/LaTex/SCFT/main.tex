% !Mode:: "TeX:UTF-8"
\documentclass[12pt,a4paper]{article}
\input{en_preamble.tex}
\input{xecjk_preamble.tex}

\title{优化区域形状的SCFT模拟}
\author{作者:周铁军}
\date{\chntoday}
\begin{document}
\maketitle
\newpage


\section{问题简述}
\subsection{引言}
考虑了$n AB$二嵌段共聚物与N在S表面聚合的总表面积$|S|$,A块的体积分数是$f$, B块的体积分数是$1 - f$,两个嵌段不同点在于其Flory-Huggins参数χN。

考虑一般曲面的SCFT问题,假设统计段的长度和体积两个块是相等的,即$ b_A = b_B = b, v_A = v_b = v_0$。共聚物链的特征长度可由旋转的无扰动半径定义,使所有空间长度都以单位表示$Rg = b\sqrt{N}/6$。在不可压缩熔体假设下,平均段密度在空间上是均匀的,由$v0 = 1/ρ0 = V/(nN)$给出。

哈密顿量:

\begin{equation}
	H\left[w_{+}, w_{-}\right]=\frac{1}{|S|} \int d \mathbf{x}\left\{-w_{+}(\mathbf{x})+\frac{w_{-}^{2}(\mathbf{x})}{\chi N}\right\}-\log Q\left[w_{+}(\mathbf{x}), w_{-}(\mathbf{x})\right]
\end{equation}

场$w\pm$满足:
\begin{equation}
	\begin{array}{l}{\frac{\partial}{\partial t} w_{+}(\mathbf{x}, t)=\frac{\delta H\left[w_{+}, w_{-}\right]}{\delta w_{+}(x, t)}} \\ {\frac{\partial}{\partial t} w_{-}(x, t)=-\frac{\delta H\left[w_{+}, w_{-}\right]}{\delta w_{-}(x, t)}}\end{array}
\end{equation}

Q是受外加场w+和w -的作用单链配分泛函:
\begin{equation}
Q=\frac{1}{|\mathcal{S}|} \int \mathrm{d} \mathbf{x} q(\mathbf{x}, 1)=\frac{1}{|\mathcal{S}|} \int \mathrm{d} \mathbf{x} q(\mathbf{x}, s) q^{\dagger}(\mathbf{x}, s), \quad \forall s \in[0,1]
\end{equation}

$\phi_A,\phi_B$为块A,B的单体密度:
\begin{equation}
\begin{aligned} \phi_{A}(\mathbf{x}) &=\frac{1}{Q} \int_{0}^{f} \mathrm{d} s q(\mathbf{x}, s) q^{\dagger}(\mathbf{x}, s) \\ \phi_{B}(\mathbf{x}) &=\frac{1}{Q} \int_{f}^{1} \mathrm{d} s q(\mathbf{x}, s) q^{\dagger}(\mathbf{x}, s) \end{aligned}
\end{equation}

\subsection{SCFT方程优化问题}
为获得给定有序结构的最优表面尺寸,将SCFT的有效哈密顿量视为曲面大小函数,以及场函数的泛函。

此外,我们用$\mathcal{S}_{\Gamma}=\left\{\Gamma \cdot \mathbf{x} : \mathbf{x} \in \mathcal{S}_{0}\right\}$代替曲面S,故完整的求解SCFT方程的优化问题变为求:
\begin{equation}
\underset{\Gamma}{\min } \max _{w_{+}} \min _{w_{-}} H\left[w_{+}(\mathbf{x}), w_{-}(\mathbf{x}), \Gamma\right]
\end{equation}

其中$\Gamma$>0是描述表面尺寸的尺度因子。例如,球体的$\Gamma$参数是它的半径。

\subsubsection{问题求解:SCFT迭代}

1.给参数χN, f,区域S,和合适的初始分布的场w±。

2.固定S,通过$\red{FSP\text{法}}$找到SCFT的鞍点,得到有效的哈密顿量。

3.固定w±,利用曲面自适应优化方法 优化区域S, 并评估有效哈密顿量的值。

4.重复步骤2-3,直到有效哈密顿量差异小于给定的收敛准则。

\subsubsection{鞍点搜索:FSP法}

1.初始化场w±(x, 0)。

2.\red{计算一般曲面上的正向和反向传播算子q(x, s)和q†(x, s)}。

3.得到Q,φA(x)和φB(x)的积分方程,并评估有效哈密顿量H的值。

4.通过(2)式使用鞍点搜索迭代方法,更新场w+(x,t)和w-(x,t)。

5.重复步骤2-4,直到满足收敛准则。

\subsubsection{获得传播子:表面有限元离散方法}

给出如下方程,q(x,s)为正向传播子:
\begin{equation}
\begin{aligned} \frac{\partial}{\partial s} q(\mathbf{x}, s) &=\left[\Delta_{\mathcal{S}}-w(\mathbf{x}, s)\right] q(\mathbf{x}, s) \\ q(\mathbf{x}, 0) &=1 \\ w(\mathbf{x}, s) &=\left\{\begin{array}{ll}{w_{+}(\mathbf{x})-w_{-}(\mathbf{x}),} & {0 \leq s \leq f} \\ {w_{+}(\mathbf{x})+w_{-}(\mathbf{x}),} & {f \leq s \leq 1}\end{array}\right.\end{aligned}
\end{equation}

同样,有反向传播子q†(x,s),求解与q(x,s)类似。

$\textbf{q(x,s)求解过程:}$

1.将(4)式改写为变分问题
\begin{equation}
\left(\frac{\partial}{\partial s} q, v\right)_{s}=-\left(\nabla_{s} q, \nabla_{s} v\right)_{s}-(w q, v)_{s}, \text { for all } v \in H^{1}(\mathcal{S})
\end{equation}

2.用有限维空间$V_h$代替无限维空间$H^1(S)$,得到线性有限元离散化,
\begin{equation}
\left(\frac{\partial}{\partial s} q_{k}, v_{h}\right)_{S_{h}}=-\left(\nabla_{S_{h}} q_{h}, \nabla_{S_{h}} v_{h}\right)_{S_{h}}-\left(w_{h} q_{h}, v_{h}\right)_{S_{h}}, \quad \text { for all } v_{h} \in \mathcal{V}_{h}
\end{equation}
其中$q_{t}=\sum_{t=1}^{N} q_{i}(s) \varphi_{i}(x)$,$w_{A}(x, s)=\sum_{b=1}^{N} w\left(x_{i}, s\right) \varphi_{i}(x)$为$w(x,s)$的线性插值。

得到:
\begin{equation}
M \frac{\partial}{\partial s} q(s)=-(A+F) q(s)
\end{equation}

其中
\begin{equation*}
q(s)=\left(q_{1}(s), q_{2}(s), \cdots, q_{N}(s)\right)^{t}
\end{equation*}
$$
M_{i, j}=\left(\varphi_{i}, \varphi_{j}\right), A_{i, j}=\left(\nabla_{S} \varphi_{i}, \nabla_{\mathcal{S}} \varphi_{j}\right), F_{i, j}=\left(w_{h} \varphi_{i}, \varphi_{j}\right)
$$

3.使用$Crank\_Nicolson$方法离散(7)式,得:
\begin{equation}
M \frac{q^{n+1}-q^{n}}{\Delta s}=-\frac{1}{2}(A+F)\left[q^{n+1}+q^{n}\right]
\end{equation}

整理得到迭代格式:
\begin{equation}
\left[M+\frac{\Delta s}{2}(A+F)\right] q^{n+1}=\left[M-\frac{\Delta s}{2}(A+F)\right] q^{n}
\end{equation}

\newpage
\section{计算结果}

\begin{table}[h]
	\centering     %插入的图片居中表示  
	\caption{计算参数值设定}  
	\begin{tabular*}{10cm}{lllllll}  
		\hline  
		Test &L  & l  & h   & node  & fa  & chiAB\\  
		\hline  
		1    &10 & 2  & 0.2 & 11215 & 0.5 & 0.25 \\  
		2    &15 & 2  & 0.2 & 25190 & 0.5 & 0.25 \\
		3    &10 & 2  & 0.2 & 11215 & 0.4 & 0.25 \\
		4    &10 & 2  & 0.2 & 11215 & 0.3 & 0.25 \\ 
		\hline  
	\end{tabular*}  
\end{table}  

下面是以上不同参数条件下的计算结果图像
\begin{figure}[h]
	\begin{minipage}[t]{0.4\linewidth}%并排放两张图片,每张占行的0.4,下同 
		\centering     %插入的图片居中表示
		\includegraphics[width=1.2\textwidth]{./figures/01.png}
		\caption{Test1 第3000步迭代图像.}%图片的名称
		\label{fig:liuchengtu1}%标签,用作
	\end{minipage} 
	\hfill
	\begin{minipage}[t]{0.4\linewidth}
		\centering
		\includegraphics[width=1.2\textwidth]{./figures/02.png}
		\caption{Test2 第1000步迭代图像.}%图片的名称
		\label{fig:liuchengtu2}
	\end{minipage}
    \hfill
    \begin{minipage}[t]{0.4\linewidth}
    	\centering
    	\includegraphics[width=1.2\textwidth]{./figures/03.png}
    	\caption{Test3 第1000步迭代图像.}%图片的名称
    	\label{fig:liuchengtu2}
    \end{minipage}
    \hfill
    \begin{minipage}[t]{0.4\linewidth}
    	\centering
    	\includegraphics[width=1.2\textwidth]{./figures/04.png}
    	\caption{Test4 第495步迭代图像.}%图片的名称
    	\label{fig:liuchengtu2}
    \end{minipage}
\end{figure}

\newpage
\section{待解决的问题}
1.如何自适应地改变区域,从而重新网格剖分?

如下图所示,首先我们可以手动的改变控制顶点坐标,从而达到目的,但是,当区域复杂或者控制顶点更多时,手动操作则不便于处理,于是我们希望能够自适应地改变区域形状,以达到预期目的。
\begin{figure}[ht]
\centering     %插入的图片居中表示
\begin{minipage}[t]{0.4\linewidth}%并排放两张图片,每张占行的0.4,下同
\includegraphics[width=1.2\textwidth]{./figures/figure1.png}
\caption{原始区域.}%图片的名称
\label{fig:liuchengtu1}%标签,用作
\end{minipage} 
\hfill
\begin{minipage}[t]{0.4\linewidth}
\centering
\includegraphics[width=1.2\textwidth]{./figures/figure2.png}
\caption{调整后区域.}%图片的名称
\label{fig:liuchengtu2}
\end{minipage}
\end{figure}


\section{附录}

\subsection{网格剖分函数}

外矩形边长: L

内矩形边长:l  

获取区域函数fd:ddiff(drectangle(p,-L,L,-L,L),drectangle(p,-l,l,-l,l))

网格剖分函数:[p,t]=distmesh2d(fd,fh,h,bbox,pfix]); \%\%fh控制网格类型,h控制网格大小,bbox为最大边界,pfix为控制顶点位置。

有限元网格剖分方法参见:http://persson.berkeley.edu/distmesh/index.html


\subsection{FEniCS使用}

1.Docker安装:

Sudo apt-get update

Sudo apt-get install docker-ce

2.运行FEniCS Docker镜像:

Curl  -s https://get.fenicsproject.org | bash

3.检验:

Sudo docker run hello-world

4.设置镜像:

Sudo docker pull quay.io/fenicsproject/stable:latest

5.启动fenics:

Sudo docker run -ti quay.io/fenicsproject/stable:latest

6.主机与容器文件共享:

Sudo docker run -ti -v\$(pwd):/home/fenics/shared quay.io/fenicsproject/stable:latest
\end{document}
