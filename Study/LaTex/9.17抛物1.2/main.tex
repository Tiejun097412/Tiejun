% !Mode:: "TeX:UTF-8"
\documentclass[12pt,a4paper]{article}
\input{en_preamble.tex}
\input{xecjk_preamble.tex}

\title{求解抛物方程1.2}
\author{作者:周铁军}
\date{\chntoday}
\begin{document}
\maketitle
\newpage
\section{抛物方程}
 题目:
 
 \begin{equation}
  \begin{array}{l}{ u_t = \alpha(u_{xx}+u_{yy})+f(x,t) \qquad (x,y) \in G=(0,1)\times(0,1),t > 0}\\
  {u(0,y,t) = u(1,y,t)=0,\qquad y \in [0,1],t \geq 0 }\\
  {u(x,0,t) = u(x,1,t)=0,\qquad x \in [0,1],t \geq 0}\\
  {u(x,y,0)=0,\qquad (x,y) \in [0,1] \times [0,1]}\end{array}
 \end{equation}
其中$f(x,y,t) = sin{5 \pi t}sin{2 \pi x}sin{\pi y},\alpha = 1$,网格步长$h_1 = h_2 = 0.1,0.05,\tau =0.01$.计算$u$在$t=0.1,0.2,0.4,0.8$的近似值。

\section{ADI法求解}
首先,对进行网格剖分,其中$h_1$和$h_2$分别为区间$[0,1]\times[0,1]$上的剖分步长。

其次,利用五点差分离散二阶空间导数$\frac{\partial^2 u}{\partial x^2}$和$\frac{\partial^2 u}{\partial y^2}$,得到:
\begin{equation}
\begin{array}{l}{\frac{\partial^2 u}{\partial x^2} = \frac{u_{i+1,j}+u_{i-1,j}-2u_{i,j}}{h_1^2} + O(h_1^2)}\\
{\frac{\partial^2 u}{\partial y^2} = \frac{u_{i,j+1}+u_{i,j-1}-2u_{i,j}}{h_2^2}+ O(h_2^2)}\end{array}
\end{equation}
利用向前差分离散一阶时间导数$\frac{\partial u}{\partial t}$,得到:
\begin{equation}
\begin{array}{l}{\frac{\partial u}{\partial t} = \frac{u^{n+1}_{i,j}-u^n_{i,j}}{\tau} + O(\tau)}\end{array}
\end{equation}
其中$i,j=0,1,...,n$,于是方程可表示为:
$$\frac{u^{n+\frac{1}{2}}_{i,j}-u^n_{i,j}}{\frac{\tau}{2}}=\frac{u^{n+\frac{1}{2}}_{i+1,j}+u^{n+\frac{1}{2}}_{i-1,j}-2u^{n+\frac{1}{2}}_{i,j}}{h^2}+\frac{u^n_{i,j+1}+u^n_{i,j-1}-2u^n_{i,j}}{h^2} + f^n(x(i),y(j)),$$

$$\frac{u^{n+1}_{i,j}-u^{n+\frac{1}{2}}_{i,j}}{\frac{\tau}{2}}=\frac{u^{n+\frac{1}{2}}_{i+1,j}+u^{n+\frac{1}{2}}_{i-1,j}-2u^{n+\frac{1}{2}}_{i,j}}{h^2}+\frac{u^{n+1}_{i,j+1}+u^{n+1}_{i,j-1}-2u^{n+1}_{i,j}}{h^2} + f^n(x(i),y(j)),$$

化简得:
$$-\frac{\tau}{2h^2}(u^{n+\frac{1}{2}}_{i+1,j}+u^{n+\frac{1}{2}}_{i-1,j})+(1+\frac{\tau}{h^2})u^{n+\frac{1}{2}}_{i,j} = \frac{\tau}{2h^2}(u^n_{i,j+1}+u^n_{i,j-1})+(1-\frac{\tau}{h^2})2u^n_{i,j} + \frac{\tau}{2}f^n(x(i),y(j)),$$

$$-\frac{\tau}{2h^2}(u^{n+1}_{i,j+1}+u^{n+1}_{i,j-1})+(1+\frac{\tau}{h^2})u^{n+1}_{i,j} =\frac{\tau}{2h^2}(u^{n+\frac{1}{2}}_{i+1,j}+u^{n+\frac{1}{2}}_{i-1,j})+(1-\frac{\tau}{h^2})u^{n+\frac{1}{2}}_{i,j}+\frac{\tau}{2} f^n(x(i),y(j)),$$

考虑边值条件,有:
\begin{equation}
\begin{array}{l}{u(x(0),y,t) = u(x(n),y,t)=0}\\
{u(x,y(0),t) = u(x,y(n),t)=0}\end{array}
\end{equation}
\newpage
则只需求解红色区域内的点(见图1):
\begin{figure}[ht]
	\centering
	\includegraphics[scale=0.4]{./figures/zuobiao.png}
	\caption{网格剖分}
	\label{fig:label}	
\end{figure}


令$U^k = (u^k_{11},u^k_{12},...,u^k_{1,n-1},u^k_{21},u^k_{22},...,u^k_{2,n-1},...)^{T} \quad (U^k\text{为}(n-1)^2 \times 1 \text{向量})$

$F^k = (f^k_{11},f^k_{12},...,f^k_{1,n-1},f^k_{21},f^k_{22},...,f^k_{2,n-1},...)^{T} \quad (F^k\text{为}(n-1)^2 \times 1\text{向量},f^k_{i,j}=f(x(i),y(j),t(k)) = sin{5 \pi t_k}sin{2 \pi x_i}sin{\pi y_j})$,

系数矩阵:

\begin{equation}
A_1 = \left[                 
\begin{array}{cccc}
B_1& C &\cdots  & 0\\
C& B_1 &\cdots& 0\\
\vdots  & \vdots & \ddots & \vdots \\
0& 0 &\cdots  &B_1
\end{array}
\right ]_{(n-1)^2 \times (n-1)^2},
\end{equation}
其中,
\begin{equation}
 B_1 =(1+\frac{\tau}{h^2}) I_{n-1}  ,\quad C = -\frac{\tau}{2h^2} I_{n-1}
\end{equation}
\newpage
\begin{equation}
A_2 = \left[                 
\begin{array}{cccc}
B_2& 0 &\cdots  & 0\\
0& B_2 &\cdots& 0\\
\vdots  & \vdots & \ddots & \vdots \\
0& 0 &\cdots  &B_2
\end{array}
\right ]_{(n-1)^2 \times (n-1)^2},\quad
B_2 = \left[                 
\begin{array}{cccc}
1-\frac{\tau}{h^2}& \frac{\tau}{2h^2} &\cdots  & 0\\
\frac{\tau}{2h^2}& 1-\frac{\tau}{h^2} &\cdots& 0\\
\vdots  & \vdots & \ddots & \vdots \\
0& 0 &\cdots  &1-\frac{\tau}{h^2}
\end{array}
\right ]_{(n-1) \times (n-1)}
\end{equation}

\begin{equation}
A_3 = \left[                 
\begin{array}{cccc}
B_3& 0 &\cdots  & 0\\
0& B_3 &\cdots& 0\\
\vdots  & \vdots & \ddots & \vdots \\
0& 0 &\cdots  &B_3
\end{array}
\right ]_{(n-1)^2 \times (n-1)^2},\quad
B_3 = \left[                 
\begin{array}{cccc}
1+\frac{\tau}{h^2}& -\frac{\tau}{2h^2} &\cdots  & 0\\
-\frac{\tau}{2h^2}& 1+\frac{\tau}{h^2} &\cdots& 0\\
\vdots  & \vdots & \ddots & \vdots \\
0& 0 &\cdots  &1+\frac{\tau}{h^2}
\end{array}
\right ]_{(n-1) \times (n-1)}
\end{equation}


\begin{equation}
A_4 = \left[                 
\begin{array}{cccc}
B_4& -C &\cdots  & 0\\
-C& B_4 &\cdots& 0\\
\vdots  & \vdots & \ddots & \vdots \\
0& 0 &\cdots  &B_4
\end{array}
\right ]_{(n-1)^2 \times (n-1)^2},\quad
B_4 =(1-\frac{\tau}{h^2}) I_{n-1}  
\end{equation}

于是得到如下迭代格式:
$$A_1 U^{n+\frac{1}{2}} = A_2 U^{n} + \frac{\tau}{2} f^n(x(i),y(j)) $$

$$A_3 U^{n+1} = A_4 U^{n+\frac{1}{2}} + \frac{\tau}{2} f^{n+\frac{1}{2}}(x(i),y(j)) $$
\section{求解结果}
不妨令$tmax = 1$,则$t \in [0,tmax]$,

Part I:$h_1 = h_2 = 0.05$不同时刻的u的近似值的图像。

\begin{figure}[ht]
	\begin{minipage}[t]{0.4\linewidth}%并排放两张图片,每张占行的0.4,下同 
		\centering     %插入的图片居中表示
		\includegraphics[width=1.2\textwidth]{./figures/time01.png}
		\caption{0.1时刻的u的近似解.}%图片的名称
		\label{fig:liuchengtu1}%标签,用作
	\end{minipage} 
	\hfill
	\begin{minipage}[t]{0.4\linewidth}
		\centering
		\includegraphics[width=1.2\textwidth]{./figures/time02.png}
		\caption{0.2时刻的u的近似解.}%图片的名称
		\label{fig:liuchengtu2}
	\end{minipage}
\end{figure}

\begin{figure}[ht]
	\begin{minipage}[t]{0.4\linewidth}%并排放两张图片,每张占行的0.4,下同 
		\centering     %插入的图片居中表示
		\includegraphics[width=1.2\textwidth]{./figures/time04.png}
		\caption{0.4时刻的u的近似解.}%图片的名称
		\label{fig:liuchengtu1}%标签,用作
	\end{minipage} 
	\hfill
	\begin{minipage}[t]{0.4\linewidth}
		\centering
		\includegraphics[width=1.2\textwidth]{./figures/time08.png}
		\caption{0.8时刻的u的近似解.}%图片的名称
		\label{fig:liuchengtu2}
	\end{minipage}
\end{figure}
\newpage
Part II:$h_1 = h_2 = 0.1$不同时刻的u的近似值的图像。
\begin{figure}[ht]
	\begin{minipage}[t]{0.4\linewidth}%并排放两张图片,每张占行的0.4,下同 
		\centering     %插入的图片居中表示
		\includegraphics[width=1.2\textwidth]{./figures/TIME01.png}
		\caption{0.1时刻的u的近似解.}%图片的名称
		\label{fig:liuchengtu1}%标签,用作
	\end{minipage} 
	\hfill
	\begin{minipage}[t]{0.4\linewidth}
		\centering
		\includegraphics[width=1.2\textwidth]{./figures/TIME02.png}
		\caption{0.2时刻的u的近似解.}%图片的名称
		\label{fig:liuchengtu2}
	\end{minipage}
\end{figure}
\begin{figure}[ht]
	\begin{minipage}[t]{0.4\linewidth}%并排放两张图片,每张占行的0.4,下同 
		\centering     %插入的图片居中表示
		\includegraphics[width=1.2\textwidth]{./figures/TIME04.png}
		\caption{0.4时刻的u的近似解.}%图片的名称
		\label{fig:liuchengtu1}%标签,用作
	\end{minipage} 
	\hfill
	\begin{minipage}[t]{0.4\linewidth}
		\centering
		\includegraphics[width=1.2\textwidth]{./figures/TIME08.png}
		\caption{0.8时刻的u的近似解.}%图片的名称
		\label{fig:liuchengtu2}
	\end{minipage}
\end{figure}
\newpage
以上为$u$在$t=0.1,0.2,0.4,0.8$的近似值。

这一次可以看到,随着迭代次数的不断增加,函数近似值的图像最终区域稳定。
\newpage
\section{误差分析}
我们选取如下节点的近似解来分析误差
\begin{equation}
\left[                 
\begin{array}{cccc}
(0.2,0.2)& (0.4,0.2) &\cdots  & (0.8,0.2)\\
(0.2,0.4)& (0.4,0.4) &\cdots& (0.8,0.4)\\
\vdots  & \vdots & \ddots & \vdots \\
(0.2,0.8)& (0.4,0.8) &\cdots  &(0.8,0.8)
\end{array}
\right]
\end{equation}

如图,"$\times$"表示的位置
\begin{figure}[ht]
	\centering
	\includegraphics[scale=0.1]{./figures/jiedian.jpg}
	\caption{节点坐标}
	\label{fig:label}	
\end{figure}

作如下四种剖分,剖分步长h = 0.2,0.1,0.05,0.025;

并将剖分最密者(即 h = 0.025)看作精确解。

得到如下误差
\begin{table*}
	\centering  
	\caption{误差分析}  
	\begin{tabular*}{12cm}{lllll}  
		\hline  
		h & max范数  & $L_2$范数 &max的阶& $L_2$的阶\\  
		\hline  
		0.2  & 1.5467e-4 & 4.2750e-4 & 2.1711 & 2.1711\\  
		0.1  & 3.4344e-5 & 9.4925e-5 & 2.3470 & 2.3470\\
		0.05  & 6.7503e-6 & 1.8657e-5& \ & \ \\
		0.025  &   &看作精确解   & & \\  
		\hline  
	\end{tabular*}  
\end{table*}

\end{document}
