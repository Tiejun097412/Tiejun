% !Mode:: "TeX:UTF-8"
\documentclass[12pt,a4paper]{article}
\input{en_preamble.tex}
\input{xecjk_preamble.tex}

\title{求解抛物方程(1.3)}
\author{作者:周铁军}
\date{\chntoday}
\begin{document}
\maketitle
\newpage
\section{抛物方程}
 题目:
 
 \begin{equation}
  \begin{array}{l}{ u_t = \frac{1}{4}(u_{xx}+u_{yy}) \qquad (x,y) \in G=(0,1)\times(0,1),t > 0}\\
  {u(0,y,t) = u(1,y,t)=0,\qquad y \in (0,1),t \geq 0 }\\
  {u_y(x,0,t) = u_y(x,1,t)=0,\qquad x \in (0,1),t \geq 0}\\
  {u(x,y,0)=sin{\pi x}cos{\pi y}}\end{array}
 \end{equation}
其中精确解为$u = sin{\pi x}cos{ \pi y}exp{(-\frac{\pi}{8}t)} $,要求验证误差阶。

\subsection{求解}
首先,对进行网格剖分,其中$h_1$和$h_2$分别为区间$x \in (0,1)$和$y \in (0,1)$上的剖分步长。

其次,利用五点差分离散二阶空间导数$\frac{\partial^2 u}{\partial x^2}$和$\frac{\partial^2 u}{\partial y^2}$,得到:
\begin{equation}
\begin{array}{l}{\frac{\partial^2 u}{\partial x^2} = \frac{u_{i+1,j}+u_{i-1,j}-2u_{i,j}}{h_1^2} + O(h_1^2)}\\
{\frac{\partial^2 u}{\partial y^2} = \frac{u_{i,j+1}+u_{i,j-1}-2u_{i,j}}{h_2^2}+ O(h_2^2)}\end{array}
\end{equation}
利用向前差分离散一阶时间导数$\frac{\partial u}{\partial t}$,得到:
\begin{equation}
\begin{array}{l}{\frac{\partial u}{\partial t} = \frac{u^{n+1}_{i,j}-u^n_{i,j}}{\tau} + O(\tau)}\end{array}
\end{equation}
其中$i,j=0,1,...,n$,于是方程可表示为:
$$\frac{u^{n+1}_{i,j}-u^n_{i,j}}{\tau}=\frac{1}{4}(\frac{u^n_{i+1,j}+u^n_{i-1,j}-2u^n_{i,j}}{h_1^2}+\frac{u^n_{i,j+1}+u^n_{i,j-1}-2u^n_{i,j}}{h_2^2}),$$

化简得:
$$u^{n+1}_{i,j} = (1-\frac{ \tau}{2h_1^2}-\frac{ \tau}{2h_2^2})u^n_{i,j}-\frac{\tau}{4h_1^2}(u^n_{i+1,j}+u^n_{i-1,j})-\frac{\tau}{4h_2^2}(u^n_{i,j+1}+u^n_{i,j-1}),$$

考虑边值条件,有:
\begin{equation}
\begin{array}{l}{u(x(0),y,t) = u(x(n),y,t)=0}\\
{u_y(x,y(0),t) = u_y(x,y(n),t)=0}\end{array}
\end{equation}


化为矩阵形式:
$$U^{n+1} = A U^{n}$$

其中$U^k = (u^k_{11},u^k_{12},...,u^k_{1,n-1},u^k_{21},u^k_{22},...,u^k_{2,n-1},...)^{T} \quad (U^k\text{为}(n-1)^2 \times 1 \text{向量})$

$F^k = (f^k_{11},f^k_{12},...,f^k_{1,n-1},f^k_{21},f^k_{22},...,f^k_{2,n-1},...)^{T} \quad (F^k\text{为}(n-1)^2 \times 1\text{向量},f^k_{i,j}=f(x(i),y(j),t(k)) = sin{5 \pi t_k}sin{2 \pi x_i}sin{\pi y_j})$,

系数矩阵:

\begin{equation}
A = \left[                 
\begin{array}{cccc}
B& C &\cdots  & 0\\
C& B &\cdots& 0\\
\vdots  & \vdots & \ddots & \vdots \\
0& 0 &\cdots  &B
\end{array}
\right ]_{(n-1)^2 \times (n-1)^2},
\end{equation}
其中,
\begin{equation}
 B = \left[                 
\begin{array}{cccc}
1-\frac{\tau}{2h_1^2}-\frac{\tau}{2h_2^2}& \frac{\tau}{4h_2^2} &\cdots  & 0\\
\frac{\tau}{4h_2^2}& 1-\frac{\tau}{2h_1^2}-\frac{\tau}{2h_2^2} &\cdots& 0\\
\vdots  & \vdots & \ddots & \vdots \\
0& 0 &\cdots  &1-\frac{\tau}{2h_1^2}-\frac{\tau}{2h_2^2}
\end{array}
\right ]_{(n-1) \times (n-1)} ,\quad C = -\frac{\tau}{4h_1^2} I_{n-1}
\end{equation}
\newpage
\subsection{求解}
不妨令时间步长$\tau = 0.01$,$tmax = 1$,则$t \in [0,tmax]$,

下面是不同剖分下,最后一步迭代u的近似值与精确解的误差的图像。

\begin{figure}[ht]
	\begin{minipage}[t]{0.4\linewidth}%并排放两张图片,每张占行的0.4,下同 
		\centering     %插入的图片居中表示
		\includegraphics[width=1.2\textwidth]{./figures/time1.png}
		\caption{0.1时刻的u的近似解.}%图片的名称
		\label{fig:liuchengtu1}%标签,用作
	\end{minipage} 
	\hfill
	\begin{minipage}[t]{0.4\linewidth}
		\centering
		\includegraphics[width=1.2\textwidth]{./figures/time2.png}
		\caption{0.2时刻的u的近似解.}%图片的名称
		\label{fig:liuchengtu2}
	\end{minipage}
\end{figure}

\begin{figure}[ht]
	\begin{minipage}[t]{0.4\linewidth}%并排放两张图片,每张占行的0.4,下同 
		\centering     %插入的图片居中表示
		\includegraphics[width=1.2\textwidth]{./figures/time4.png}
		\caption{0.1时刻的u的近似解.}%图片的名称
		\label{fig:liuchengtu1}%标签,用作
	\end{minipage} 
	\hfill
	\begin{minipage}[t]{0.4\linewidth}
		\centering
		\includegraphics[width=1.2\textwidth]{./figures/time8.png}
		\caption{0.2时刻的u的近似解.}%图片的名称
		\label{fig:liuchengtu2}
	\end{minipage}
\end{figure}



\subsection{问题}

由上图的结果分析,迭代过程中,数值的量级发生了剧变,我检查了系数矩阵A,以及余项F,是没有操作上的错误的。

代码见附录。

\end{document}
