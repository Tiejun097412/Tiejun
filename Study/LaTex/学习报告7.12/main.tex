% !Mode:: "TeX:UTF-8"
\documentclass[12pt,a4paper]{article}
\input{en_preamble.tex}
\input{xecjk_preamble.tex}

\title{学习报告}
\author{作者:周铁军}
\date{\chntoday}
\begin{document}
\maketitle
\newpage
\section{显式欧拉法}
欧拉法是常微分方程的数值解法的一种,其基本思想是迭代。其中分为前进的EULER法、后退的EULER法、改进的EULER法。所谓迭代,就是逐次替代,最后求出所要求的解,并达到一定的精度。

{\bfseries 向前欧拉}形式如下:
$$u_{n+1}=u_{n}+h f\left(u_{n}, t_{n}\right)$$
$\text{初始条件:}u(t_0)=u_0$

{\bfseries 向后欧拉}形式如下:
$$u_{n+1}=u_{n}+h f\left(u_{n+1}, t_{n+1}\right)$$
$\text{初始条件:}u(t_0)=u_0$

\subsection{推导过程}
对于常微分方程:$$\frac{du}{dx}=f(x,u)$$
$$u(x_0)=u_0$$

离散变量,用差商近似代替导数:
$$\frac{du}{dx} \approx \frac{u(x_{i+1})-u(x_{i})}{\delta x} \qquad x_i\text{为任意一点,\qquad}(i=0,1,2,...)$$

于是有$$u(x_{i+1})=u(x_{i})+\delta x *f(x_i,u(x_i))$$


\begin{table}  
	%\caption{设置表格总长}  
	\begin{tabular*}{16cm}{ll}  
		\hline  
		Python代码 \\  
		\hline  
		&import numpy as np\\
		&import matplotlib.pyplot as plt\\	
		&N=10000\\
		&x=np.linspace(0,1,N+1)\\
		&u=np.zeros(N+1)\\
		&u[0]=1\\
		&uexact=np.exp(x)\\
		&step=1/N\\
		
		&def eular(x,u):\\
		&\qquad return u\\
		
		&for i in range(0,N):\\
		&\qquad[i+1] = u[i] + eular(x[i],u[i])*step\\
		
		&plt.plot(x,u-uexact)\\
		\hline  
	\end{tabular*}  
\end{table} 


\begin{figure}[ht]
结果图像如下:\\
	\centering
	\includegraphics[scale=0.6]{./figures/Figure_6.png}
	\caption{this is the error}
	\label{fig:label}
\end{figure}

\newpage
\section{B样条插值}
\subsection{德布尔-考克斯递推公式}
B样条有多种等价定义,这里用到的是德布尔-考克斯递推公式,
\begin{figure}[ht]
	\centering
	\includegraphics[scale=0.6]{./figures/Figure_7.png}
	\caption{ }
	\label{fig:label}
\end{figure}

$N_{i,k}(u)$表示k次样条基函数,下标i表示序号。

易知要确定第i个k次B样条$N_{i,k}(u)$,需要用到$u_i,u_{i+1},...,u_{i+k+1}$共$k+2$个节点。称$[u_i,u_{n+k+1}]为N_{i,k}(u)$的支撑区间


对于曲线方程中相应的$n+1$个控制顶点$d_i(i=0,1,2,...,n)$,需要用到$n+1$个k次B样条基函数,所需要的支撑区间所含节点为这组B样条基的节点矢量$U=[u_0,u_1,...,u_{n+k+1}]$

下面介绍一个案例的不同控制点的图像
\begin{figure}[ht]
	\centering
	\includegraphics[scale=0.6]{./figures/16.png}
	\caption{用16个控制点的得到的B样条函数图像}
	\label{fig:label}
\end{figure}

\begin{figure}[ht]
	\centering
	\includegraphics[scale=0.6]{./figures/80.png}
	\caption{用80个控制点的得到的B样条函数图像}
	\label{fig:label}
\end{figure}


\end{document}
